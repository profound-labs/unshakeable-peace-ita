
{\hspace*{17pt}
\begin{minipage}{0.9\linewidth}
\small\itshape
Il seguente discorso fu informalmente offerto dal Ven. Ajahn Chah a un
monaco studioso, venuto a rendergli omaggio.
\end{minipage}
}

\vspace*{0.8\baselineskip}

{\scshape La vera ragione per studiare il dhamma}, gli insegnamenti del Buddha, è
quella di trovare un modo per trascendere la sofferenza e realizzare la
pace e la felicità. Sia che studiamo i fenomeni fisici o mentali, la
mente (\emph{citta}) o i fattori psicologici (\emph{cetasika}), siamo
sulla retta via solo quando poniamo la liberazione dalla sofferenza come
il nostro scopo ultimo. La sofferenza esiste per sue precise cause e
condizioni.

Cercate di capire che la mente, quando è tranquilla, si trova nel suo
stato naturale, normale. Appena si muove, diventa condizionata
(\emph{saṅkhāra}). Quando la mente è attratta da qualcosa, diventa
condizionata. Quando sorge l'avversione, diventa condizionata. Il
desiderio di muoversi qua e là nasce dal condizionamento. Se la nostra
consapevolezza non riesce a tenere il passo con queste proliferazioni
mentali man mano che nascono, la mente si metterà a inseguirle e ne sarà
condizionata. Ogni volta che la mente si muove, in quel momento preciso,
diventa una realtà convenzionale.

Perciò il Buddha ci ha detto di contemplare queste fluttuanti condizioni
mentali. Ogni volta che si muove, la mente diventa instabile e
impermanente (\emph{anicca}), insoddisfacente (\emph{dukkha}) e non può
essere considerata un sé (\emph{anatta}). Queste sono le tre
caratteristiche universali di ogni fenomeno condizionato. Il Buddha ci
ha insegnato ad osservare e contemplare questi movimenti della mente.

La stessa cosa vale per l'insegnamento dell'origine dipendente
(\emph{paṭicca-samuppāda}): la comprensione errata (\emph{avijja}) è la
causa e condizione del sorgere delle formazioni volitive kammiche
(\emph{saṅkhāra)} che sono la causa e condizione del sorgere della
coscienza (\emph{viññaṇa}) che è la causa e condizione del sorgere di
mente e materia (\emph{nāma-rūpa}) e così via per tutta la sequenza che
abbiamo studiato nelle Scritture. Il Buddha ha separato ogni anello
della catena per rendere più facile lo studio. E' un'accurata
descrizione della realtà, ma quando questo processo avviene per davvero
nella vita reale, gli studiosi non sono in grado di tener dietro a
quello che succede. E' come cadere dalla cima di un albero e schiantarsi
al suolo. Non abbiamo la più pallida idea di quanti rami abbiamo passato
cadendo. Allo stesso modo, quando la mente è improvvisamente colpita da
un'impressione mentale, se è gradevole, allora si lascia trasportare dal
buon umore. La considera buona senza rendersi conto della catena di
condizioni che l'hanno resa possibile. Il processo avviene seguendo le
linee impostate teoricamente, ma contemporaneamente va oltre i limiti
della teoria stessa.

Non c'è niente che ci avverta: ``Questa è illusione. Queste sono
formazioni volitive kammiche e questa è coscienza''. Il processo non
permette allo studioso di seguire l'evento man mano che avviene come si
seguirebbero le varie voci di una lista. Sebbene il Buddha abbia
analizzato e spiegato la sequenza dei vari momenti mentali
dettagliatamente, per me corrisponde più al cadere da un albero. Mentre
precipitiamo non abbiamo la possibilità di valutare di quanti metri e
centimetri siamo caduti. Quello che sappiamo è che abbiamo toccato terra
con un tonfo e che fa male!

Lo stesso vale per la mente. Quando inciampa in qualcosa, ciò che
sentiamo è il dolore. Da dove viene tutta questa sofferenza, dolore,
angoscia e disperazione? Non viene certo da una teoria scritta in un
libro! Non c'è nessun libro in cui vengono scritti i dettagli della
nostra sofferenza. Il dolore non corrisponde alla teoria, anche se i due
viaggiano insieme sulla stessa strada. Perciò la sola erudizione non può
stare al passo con la realtà. Ed è per questo che il Buddha ci ha
insegnato a coltivare la chiara comprensione per conto nostro. Qualsiasi
cosa nasca, nasce in questa comprensione. Quando colui che conosce,
conosce secondo verità, allora la mente e i fattori psicologici vengono
riconosciuti come non nostri. Alla fine tutti questi fenomeni devono
essere abbandonati e gettati via come se fossero spazzatura. Non
dobbiamo attaccarci ad essi e tanto meno dargli importanza.

\clearpage

\Section{Teoria e realtà}

Il Buddha non ci ha insegnato a guardare la mente e i fattori mentali
perché ci attaccassimo ai concetti. La sua sola intenzione era che li
riconoscessimo come impermanenti, insoddisfacenti e senza un sé. E poi
dobbiamo lasciarli andare. Metterli da parte. Siatene consapevoli e
conosceteli nel momento che sorgono. Questa mente è già stata
condizionata. E' già stata addestrata e condizionata a girare su se
stessa e a stare lontana da uno stato di pura consapevolezza. Man mano
che gira, crea fenomeni condizionati che la influenzano ulteriormente e
in tal modo la proliferazione va avanti, producendo il bene, il male, e
ogni altra cosa sotto il sole. Il Buddha ci ha insegnato a lasciar
perdere tutto. All'inizio però dovete familiarizzarvi con la teoria in
modo che, in un secondo tempo, sarete in grado di lasciar andare ogni
cosa. E' un processo del tutto naturale. La mente è semplicemente così.
I fattori psicologici sono semplicemente così.

Prendete, ad esempio, l'Ottuplice Nobile Sentiero. Quando la saggezza
(\emph{pañña}) vede le cose correttamente attraverso l'intuizione
profonda, questa Retta Visione porta a Retta Intenzione, Retta Parola,
Retta Azione, e così via. Sono tutte condizioni psicologiche che sorgono
da questa pura conoscenza consapevole. Tale conoscenza è come una
lampada che illumina il sentiero in una notte buia. Se la conoscenza è
giusta e corrisponde alla verità, si diffonderà e illuminerà tutti gli
altri tratti del sentiero.

Qualunque cosa sperimentiamo, sorge dall'interno di questa conoscenza.
Se non esistesse questa mente, non esisterebbe neanche il conoscere.
Sono tutti fenomeni della mente. Non c'è un essere, una persona, un sé,
un noi. Non c'è un noi né un loro. Il Dhamma è semplicemente il Dhamma.
E' un processo naturale, privo di un sé. Non appartiene a noi né a
nessun altro. Non è una cosa. Qualsiasi cosa uno sperimenti fa parte di
una delle cinque categorie fondamentali (\emph{khandha}): corpo,
sensazioni, memoria/percezione, pensieri e coscienza. Il Buddha ci ha
detto di lasciar andare tutto.

La meditazione è come un bastoncino di legno. La visione profonda
(\emph{vipassana}) è un'estremità del bastoncino e la tranquillità
(\emph{samatha}) l'altra. Se prendiamo in mano un bastoncino, abbiamo
solo un'estremità o tutte e due? Quando uno raccoglie un bastone, prende
tutte e due le estremità. Qual è \emph{vipassana} e quale è \emph{samatha} allora?
Dove finisce una e comincia l'altro? Tutte e due sono la mente.
All'inizio la mente diviene tranquilla, e la pace le deriva dalla
serenità di \emph{samatha}. Concentriamo e unifichiamo la mente in stati di
pace meditativa (\emph{samādhi}). Tuttavia se la pace e l'immobilità del
\emph{samādhi} scompaiono, al suo posto sorge la sofferenza. Perché? Perché la
pace indotta dalla sola meditazione \emph{samatha} è basata sempre
sull'attaccamento. E questo attaccamento può causare sofferenza. Il
Buddha vide attraverso la propria esperienza che questo tipo di pace
mentale non era quella ultima. Le cause che sottendevano il processo di
esistenza (\emph{bhava}) non erano ancora state portate a cessazione
(\emph{nirodha}): sussistevano ancora le condizioni per la rinascita. Il
suo lavoro spirituale non era ancora completo. Perché? Perché c'era
ancora sofferenza. Perciò, partendo dalla serenità di \emph{samatha}, continuò
a contemplare, a investigare e ad analizzare la natura condizionata
della realtà fino a che si sentì libero da ogni attaccamento, anche da
quello alla tranquillità. La tranquillità è sempre parte del mondo
dell'esistenza condizionata e della realtà convenzionale. Attaccarsi a
questa pace è attaccarsi alla realtà condizionata, e finché ci
attacchiamo rimarremo impantanati nell'esistenza e nella rinascita.
Godere della pace di \emph{samatha} porta a ulteriori esistenze e rinascite.
Una volta che l'agitazione e l'irrequietezza della mente si calmano, ci
si attacca alla pace che ne risulta.

Il Buddha esaminò le cause e le condizioni che sottendono l'esistenza e
la rinascita. Fino a che non penetrò completamente il problema e non
capì la verità, continuò a sondare sempre più in profondità con la mente
calma, riflettendo su come tutte le cose, che siano calme oppure no,
vengono all'esistenza. La sua indagine fu portata avanti con decisione
fino a quando gli fu chiaro che tutto ciò che viene all'esistenza è come
un pezzo di ferro incandescente. Le cinque categorie dell'esperienza
umana (\emph{khandha}) sono pezzi di ferro incandescente. Quando il ferro
diventa incandescente, si può forse toccarlo senza bruciarsi? C'è una
parte di esso che sia fredda? Provate a toccarlo in cima, ai lati o
sotto. C'è anche una minima parte fredda? Impossibile. Quel pezzo di
ferro arroventato è tutto incandescente. Possiamo attaccarci perfino
alla tranquillità. Se ci identifichiamo con quella pace, considerando
che c'è qualcuno che è calmo e sereno, ciò rinforza il senso di un sé o
anima indipendenti. Questo senso del sé è parte della realtà
convenzionale. Pensare: ``Sono calmo, sono agitato, sono buono, sono
cattivo, sono felice o sono infelice'' ci intrappola ulteriormente
nell'esistenza e nella rinascita. E' ulteriore sofferenza. Se la
felicità svanisce, allora ci sentiamo infelici. Quando l'angoscia
svanisce, siamo di nuovo felici. Presi in questo ciclo infinito, non
facciamo che passare continuamente dal paradiso all'inferno.

Prima dell'illuminazione, il Buddha riconobbe che anche il suo cuore
seguiva questo modello di comportamento. Seppe così che non erano ancora
cessate le condizioni che lo portavano a una continua esistenza e
rinascita, quindi il suo compito non era ancora terminato.
Concentrandosi sulla condizionalità della vita, la contemplò secondo
natura: ``A causa di ciò c'è la nascita, a causa della nascita c'è la
morte e tutto il movimento di nascita e morte''.

Così il Buddha prese a
contemplare questi oggetti per poter scoprire la verità sui cinque
\emph{khandha}. Ogni cosa fisica e mentale, ogni cosa concepita e pensata,
tutto, senza eccezioni, è condizionato. Una volta compreso ciò, ci
insegnò a lasciar andare. Una volta compreso ciò, ci insegnò ad
abbandonare tutto. Ci spinse a vedere tutto alla luce di questa verità.
Se non lo facciamo, soffriremo. Non saremo in grado di lasciar andare.
Ma una volta vista la verità, riconosceremo che queste cose ci
ingannano. Come insegnò il Buddha: ``La mente non ha sostanza, non è una
cosa''.

La mente quando è nata non apparteneva a nessuno. Non muore come
qualcosa appartenente a qualcuno. E' una mente libera, radiosa e
splendente, non intrappolata in problemi o discussioni. La ragione per
cui sorgono i problemi è che la mente si fa ingannare dalle cose
condizionate, da questa percezione ingannevole di un sé. Perciò il
Buddha insegnò ad osservare questa mente. All'inizio che cosa c'è? In
verità, non c'è proprio nulla. Non sorge con le cose condizionate e non
muore con esse. Quando la mente trova qualcosa di bello non cambia per
diventare bella. Quando la mente trova qualcosa di brutto non diventa
anch'essa brutta. E' così che avviene quando si ha una chiara percezione
diretta della propria natura. C'è la comprensione che tutto è
essenzialmente privo di qualsiasi sostanza.

Con tale intuizione profonda il Buddha vide che tutto è impermanente,
insoddisfacente e senza un sé. Il Buddha vuole che anche noi
comprendiamo la stessa cosa. Allora il ``conoscere'' conoscerà secondo
verità. Pur conoscendo la felicità e l'angoscia, rimarrà impassibile.
L'emozione della felicità è una forma di nascita. La tendenza a
diventare tristi è una forma di morte. Quando c'è morte, c'è nascita, e
quello che nasce deve morire. Tutto ciò che sorge e passa è preso in
un'eterna spirale di divenire. Quando la mente del meditatori arriverà a
questo stato di comprensione, non si chiederà più se c'è un ulteriore
divenire e rinascita. E non ci sarà neanche più il bisogno di chiederlo
ad altri.

Il Buddha indagò pienamente i fenomeni condizionati e perciò fu in grado
di lasciarli andare. Furono abbandonati i cinque \emph{khandha} e il conoscere
era soltanto un'osservazione imparziale. Se sperimentava qualcosa di
positivo, non diventava positivo insieme a quello. Semplicemente
osservava e rimaneva vigile e attento. Se sperimentava qualcosa di
negativo non diventava negativo. E perché? Perché la sua mente si era
liberata da quelle cause e condizioni. Aveva penetrato la Verità. Non
esistevano più le condizioni per la rinascita. Questo è un conoscere
certo e affidabile. Questa è una mente veramente in pace. Questo è ciò
che non nasce, non si ammala, non invecchia e non muore. Non è né causa
né effetto, né dipende da causa ed effetto. E' indipendente dal processo
del condizionamento causale. Allora le cause cessano anch'esse senza
lasciare condizionamenti residui. Una mente così è oltre la nascita e la
morte, sopra e oltre la felicità e il dolore, sopra e oltre il bene e il
male. Che dire? E' al di là delle limitazioni del linguaggio che tenta
di descriverla. Tutte le condizioni di sostegno sono cessate e ogni
tentativo di descriverla non farebbe che portare all'attaccamento.
Allora le parole diventano la teoria della mente.

Le descrizioni teoriche della mente e del suo funzionamento sono
precise, ma il Buddha capì che questo tipo di conoscenza era
relativamente inutile. Intellettualmente capiamo qualcosa e ci crediamo,
ma non è di alcun beneficio. Non porta alla pace della mente. La
conoscenza del Buddha porta al lasciar andare. Ha come risultati
l'abbandono e la rinuncia. Perché è proprio questa mente che ci trascina
a immischiarci in ciò che è giusto e sbagliato. Se siamo saggi ci
lasciamo coinvolgere solo in ciò che è giusto. Se siamo sciocchi ci
lasciamo coinvolgere da ciò che è sbagliato. Una mente del genere è il
mondo, e il Beato, per esaminare le cose di questo mondo, ricorse alle
cose di questo mondo. Avendo infine conosciuto il mondo così com'è, fu
chiamato: ``Colui che comprende chiaramente il mondo''.

Sempre riguardo a \emph{samatha} e \emph{vipassana}, la cosa importante
è sviluppare questi stati nel proprio cuore. Solo coltivandoli veramente
in noi stessi sapremo cosa davvero sono. Possiamo andare a studiare
tutto ciò che i libri dicono a proposito dei fattori psicologici della
mente, ma questo tipo di conoscenza intellettuale non serve ad eliminare
concretamente il desiderio egoistico, la rabbia e l'illusione. Studiamo
solo la teoria riguardante il desiderio egoistico, la rabbia e
l'illusione, che descrive semplicemente le varie caratteristiche di
queste contaminazioni mentali: ``Il desiderio egoistico ha questo
significato; la rabbia vuol dire ciò; l'illusione si chiama così''.Se
conosciamo solo le loro qualità a livello teorico, possiamo parlarne
solo a quel livello. Li conosciamo, siamo intelligenti, ma quando questi
inquinanti appaiono in pratica nella nostra mente, corrispondono alla
teoria o no? Per esempio, quando sperimentiamo qualcosa di sgradevole,
reagiamo e diventiamo di cattivo umore? Ci attacchiamo ad esso?
Riusciamo a lasciar andare? Se sorge l'avversione e la riconosciamo,
continuiamo a rimanerci attaccati? Oppure, nel momento che la vediamo,
la lasciamo andare? Se troviamo che vediamo qualcosa che non ci piace e
tratteniamo questa avversione nel cuore, sarebbe allora meglio tornare a
studiare tutto daccapo. Perché così non va bene. La pratica non è ancora
perfetta. Quando raggiunge la perfezione, il lasciar andare avviene
semplicemente. Guardatelo sotto questa luce.

Dobbiamo guardare profondamente e veramente nei nostri cuori per poter
sperimentare i frutti di questa pratica. Cercare di spiegare la
psicologia della mente in termini di infiniti momenti di coscienza
separati e delle loro diverse caratteristiche non è, secondo me, un modo
di portare avanti la pratica. C'è ben altro da vedere. Se studiamo
queste cose, allora conosciamole a livello assoluto, con una
comprensione chiara e penetrante. Senza la chiarezza della comprensione
intuitiva, come potremmo mai venirne a capo? Non se ne vedrà mai la
fine. Non completeremo mai i nostri studi.

Perciò \emph{praticare} il Dhamma è estremamente importante. E' così che
ho studiato: praticando. Non sapevo niente di momenti-pensiero o fattori
psicologici. Osservavo solo la qualità del conoscere. Se sorgeva un
pensiero di odio, mi chiedevo il perché. Se sorgeva un pensiero d'amore,
mi chiedevo il perché. Questa è la via. Che differenza fa se lo
etichettiamo come pensiero o fattore psicologico? Penetrate il più
possibile dentro di esso fino a che sarete in grado di risolvere questa
sensazione di amore o odio, di farla svanire completamente dal cuore.
Quando riuscii a smettere di odiare e amare in ogni circostanza, fui in
grado di trascendere la sofferenza. E allora non importa quello che
succede, il cuore e la mente sono liberati e in pace. Nulla rimane.
Tutto è cessato.

Praticate in questo modo. Se gli altri vogliono parlare della teoria,
che lo facciano, sono affari loro. Ma per quanto se ne discuta, al lato
pratico tutto si riduce a quest'unico punto. Quando qualcosa nasce,
nasce proprio lì. Che sia tanto o poco, nasce sempre lì. Quando cessa,
la cessazione avviene lì. E dove altro potrebbe? Il Buddha chiamò questo
punto ``il conoscere'' e quando questo conoscerà bene le cose così come
stanno, in accordo con la verità, allora capiremo il significato della
mente. Le cose ci ingannano continuamente. Mentre le studiate, in quel
preciso momento, vi ingannano. Come altro potrei esprimermi? Anche se
sapete di cosa si tratta, vi ingannano lo stesso, proprio nel punto in
cui le conoscete. Questa è la situazione. Il problema è questo: secondo
me il Buddha non intendeva che noi conoscessimo soltanto il nome di
queste cose. Lo scopo dell'insegnamento del Buddha è trovare il modo di
liberarsi di queste cose, scoprendo le cause che le sottendono.

\Section{Sīla, samādhi e pañña}

Ho cominciato a praticare il Dhamma senza saperne molto. Sapevo solo che
la via verso la liberazione cominciava con la virtù (\emph{sīla}).
La virtù è lo splendido inizio del cammino. La profonda pace di
\emph{samādhi} ne è la parte mediana. La saggezza
(\emph{pañña}) ne è la fine eccelsa. Sebbene nella pratica li possiamo
suddividere in tre diversi momenti, man mano che li osserviamo sempre
più in profondità, vediamo che queste tre qualità si fondono in un unico
elemento. Per sostenere la virtù dobbiamo essere saggi. Alle persone di
solito consigliamo di mantenere una base etica, praticando per prima
cosa i cinque precetti, in modo che la loro moralità si consolidi.
Tuttavia, per perfezionare la virtù ci vuole molta saggezza. Dobbiamo
considerare le nostre parole e azioni, e analizzarne le conseguenze.
Questo è ciò che fa la saggezza. Dobbiamo contare sulla saggezza per
coltivare la virtù.

In teoria la virtù viene per prima, poi viene \emph{samādhi} e infine la
saggezza, ma quando le ho analizzate meglio, ho visto che la saggezza è
il fondamento di ogni altro aspetto della pratica. Per comprendere fino
in fondo le conseguenze di ciò che diciamo e facciamo -- specialmente le
conseguenze dannose -- abbiamo bisogno che la saggezza sorvegli, guidi e
analizzi il meccanismo di causa ed effetto. Ciò purificherà le parole e
le azioni. Una volta familiarizzati con i comportamenti etici e non
etici, sapremo come metterli in pratica. Abbandoniamo ciò che è male e
coltiviamo ciò che è bene. Abbandoniamo ciò che è sbagliato e coltiviamo
ciò che è giusto. Questa è la virtù. Man mano che lo facciamo, il cuore
diventa sempre più saldo e risoluto. Un cuore saldo e fermo è libero
dall'apprensione, dal rimorso e dalla confusione riguardo alle proprie
azioni e parole. Questo è \emph{samādhi}.

Questa salda unificazione della mente costituisce un'altra importante
fonte di energia nella nostra pratica di Dhamma, permettendo una
contemplazione più profonda degli oggetti, dei suoni, ecc. man mano che
li sperimentiamo. Una volta che la mente si è stabilizzata in una ferma
e salda consapevolezza ed è in pace, possiamo impegnarci in un'indagine
approfondita della realtà di corpo, sensazioni, percezioni, pensieri,
coscienza, oggetti visibili, suoni, odori, gusti, sensazioni fisiche e
oggetti mentali. Siccome tutte queste cose sorgono in continuazione, noi
li indaghiamo in continuazione con la sincera determinazione di non
perdere la consapevolezza. Sapremo così cosa sono veramente queste cose.
Sorgono seguendo una loro verità naturale. Man mano che la comprensione
aumenta, nasce la saggezza. Una volta che c'è la chiara comprensione di
come stanno veramente le cose, il nostro vecchio modo di percepire viene
sradicato e la conoscenza intellettuale si trasforma in saggezza. E'
così che la virtù, la concentrazione e la saggezza si fondono e
funzionano all'unisono.

Man mano che la saggezza cresce, impavida e forte, il \emph{samādhi} a sua
volta diventa sempre più saldo. Più il \emph{samādhi} è saldo, più la virtù
diventa incrollabile e totale. La perfezione della virtù alimenta il
\emph{samādhi} e l'aumento di vigore del \emph{samādhi} conduce alla maturazione della
saggezza. Questi tre aspetti della pratica convergono e si intersecano.
Uniti, formano l'Ottuplice Nobile Sentiero, la via del Buddha. Quando la
virtù, il \emph{samādhi} e la saggezza raggiungono il loro culmine, questo
Sentiero ha il potere di sradicare ciò che inquina la purezza della
mente.%
\pagenote{\emph{Kilesa}: contaminazioni; qualità mentali che contaminano,
inquinano, macchiano il cuore: desiderio egoistico o sensuale, rabbia,
illusione, e qualunque stato mentale basato su di essi.}
Quando sorge il desiderio sensuale, quando la
rabbia e l'ignoranza mostrano la faccia, solo il Sentiero è in grado di
eliminarne ogni traccia.

La pratica di Dhamma si svolge nel contesto delle Quattro Nobili Verità:
sofferenza (\emph{dukkha}), origine della sofferenza (\emph{samudaya}),
cessazione della sofferenza (\emph{nirodha}) e il Sentiero che conduce
alla cessazione della sofferenza (\emph{magga}). Questo Sentiero
consiste di virtù, \emph{samādhi} e saggezza, che sono il contesto entro il
quale si addestra il cuore. Il loro vero significato non lo si trova
nelle parole, ma giace profondo nel nostro cuore. E' così che sono
virtù, \emph{samādhi} e saggezza. Si alternano continuamente. Il Nobile
Ottuplice Sentiero comprende tutto ciò che sorge: ogni cosa visibile,
suono, odore, gusto, sensazione fisica o oggetto mentale. Tuttavia, se i
fattori dell'Ottuplice Nobile Sentiero sono deboli e incerti, le
contaminazioni si impadroniranno della mente. Se invece il Nobile
Sentiero è forte e coraggioso, vincerà e distruggerà le contaminazioni
inquinanti. Se gli inquinanti sono coraggiosi e potenti mentre il
Sentiero è debole e fragile, questi conquisteranno il Sentiero.
Conquisteranno il nostro cuore. Se la conoscenza non è abbastanza veloce
e pronta nel momento in cui sperimentiamo le forme, le sensazioni, le
percezioni, i pensieri, essi si impossesseranno di noi e ci
devasteranno. Il Sentiero e gli inquinanti procedono insieme. Man mano
che nel cuore si sviluppa la pratica del Dhamma, queste due forze devono
contendersi ogni passo della via. E' come se all'interno della mente ci
fossero due persone che discutono, ma in effetti sono il Sentiero del
Dhamma e gli inquinanti che si sfidano per conquistare il dominio sul
cuore. Il Sentiero alimenta e guida la nostra capacità di
contemplazione. Finché siamo in grado di contemplare correttamente, gli
inquinanti perderanno terreno. Ma se tentenniamo, lasciando che le
contaminazioni si raggruppino e riprendano forza, il Sentiero sarà
sbaragliato mentre gli inquinanti riprenderanno il dominio. Le due parti
continueranno a combattersi fino a quando non ci sarà un vincitore e la
partita sarà conclusa.

Se concentriamo i nostri sforzi per sviluppare la via del Dhamma, gli
inquinanti verranno gradualmente e costantemente sradicati. Una volta
perfettamente coltivate, le Quattro Nobili Verità prenderanno dimora nel
nostro cuore. Sotto qualsiasi forma la sofferenza si presenti, ha sempre
la sua causa. Questa è la Seconda Nobile Verità. E quale è la causa?
Virtù debole. \emph{Samādhi} debole. Saggezza debole. Quando il Sentiero non è
stabile, le contaminazioni dominano la mente. E quando esse dominano,
allora entra in gioco la Seconda Nobile Verità, che genera ogni sorta di
sofferenza. Una volta che soffriamo, scompaiono quelle qualità che
sarebbero in grado di smorzare la sofferenza. Le condizioni che fanno
sorgere il Sentiero sono virtù, \emph{samādhi} e saggezza. Quando hanno
raggiunto la loro piena maturità, il Sentiero del Dhamma non si ferma
più e avanza incessantemente per superare l'attaccamento e la bramosia
che ci riempiono di tanta angoscia. La sofferenza non può sorgere,
perché il Sentiero sta distruggendo gli inquinanti. E' a questo punto
che avviene la cessazione della sofferenza. Perché il Sentiero è in
grado di portare alla cessazione della sofferenza? Perché virtù, \emph{samādhi}
e saggezza hanno raggiunto la perfezione e il Sentiero ha preso un avvio
irrefrenabile. Tutto converge qui. Chi pratica così, secondo me, non ha
alcun bisogno di teorie sulla mente. Se la mente se ne libera, allora
diventa completamente sicura e certa. Ora, qualsiasi cammino
intraprenda, non dobbiamo pungolarla troppo perché mantenga la giusta
direzione.

Consideriamo le foglie di un albero di mango. Come sono? Basta esaminare
una singola foglia per saperlo. Anche se ce ne sono diecimila sappiamo
come sono tutte le altre soltanto guardandone una. Sono sostanzialmente
le stesse. Altrettanto si può dire per il tronco. Basta osservare il
tronco di un solo albero di mango per sapere le caratteristiche di tutti
gli altri alberi di mango. Osservatene uno solo. Tutti gli altri alberi
di mango non sono sostanzialmente diversi. Anche se ce ne fossero
centomila, quando ne conoscete uno, li conoscete tutti. Questo è ciò che
il Buddha ha insegnato.

Virtù, \emph{samādhi} e saggezza costituiscono il Sentiero del Buddha. Ma la
via non è l'essenza del Dhamma. Il Sentiero non è fine a se stesso, non
è il traguardo ultimo indicato dal Beato. È la strada che vi conduce
verso la meta. E' come la strada che avete percorso per venire da
Bangkok a questo Monastero, Wat Nong Pah Pong. Il vostro traguardo non
era la strada, ma il monastero e avevate bisogno della strada per il
viaggio. La strada su cui avete viaggiato non è il monastero. E' solo un
modo per arrivarci. Ma se volete arrivare al monastero dovete seguire la
strada. E' la stessa cosa per virtù, \emph{samādhi} e saggezza. Possiamo dire
che non sono l'essenza del Dhamma, ma che sono la strada per arrivarci.
Quando si è completamente padroni di virtù, \emph{samādhi} e saggezza, il
risultato è una profonda pace della mente. Questa è la destinazione.
Quando siamo arrivati a questa pace, anche se sentiamo un rumore, la
mente rimane tranquilla. Una volta raggiunta questa pace, non c'è altro
da fare. Il Buddha ha insegnato a lasciar andare tutto. Qualsiasi cosa
succeda, non c'è niente di cui preoccuparsi. È allora che conosceremo da
noi stessi, veramente, incontestabilmente, e non ci limiteremo più a
credere a ciò che gli altri ci dicono.

Il principio essenziale del Buddhismo è vuoto di ogni fenomeno. Non
dipende da miracolosi poteri psichici, capacità paranormali o altre cose
strane o mistiche. Il Buddha non dette importanza a queste cose. Questi
poteri esistono ed è possibile svilupparli, ma questa parte del Dhamma è
ingannevole, e per tale motivo il Buddha non le dette importanza e non
la incoraggiò. Egli lodò soltanto coloro che erano stati in grado di
liberarsi dalla sofferenza.

Tuttavia ci vuole esercizio; gli strumenti e l'attrezzatura per compiere
il lavoro sono generosità, virtù, \emph{samādhi} e saggezza. Dobbiamo usarli
per esercitarci bene. Combinati formano il Sentiero che porta verso
l'interiorità, e la saggezza è il primo passo. Questo Sentiero non può
progredire se la mente è incrostata di contaminazioni, ma se siamo
intrepidi e forti, il Sentiero eliminerà queste impurità. Però se sono
gli inquinanti ad essere intrepidi e forti distruggeranno il Sentiero.
La pratica del Dhamma semplicemente implica una incessante battaglia tra
queste due forze, fino a che si raggiungerà la fine del cammino. Esse
sono impegnate in una strenua lotta fino alla fine.

\Section{I pericoli dell'attaccamento}

Usare gli strumenti della pratica comporta fatica e sfide difficili.
Dobbiamo contare sulla pazienza, la tolleranza e la rinuncia. Dobbiamo
fare tutto da soli, sperimentare da soli, realizzare da soli. Quelli che
hanno studiato molto, tuttavia, tendono a fare un sacco di confusione.
Per esempio quando si siedono in meditazione, non appena la mente
sperimenta un pochino di tranquillità, subito cominciano a pensare:
``Ehi, questo deve essere il primo \emph{jhāna}!''%
\pagenote{\emph{Jhāna}: profonda unificazione della mente in meditazione. La vetta del \emph{samādhi}. Il Buddha ne ha insegnato otto livelli.}
E' così che lavora la loro mente. Ma questi pensieri, quando sorgono,
rovinano quella tranquillità appena sorta. Poi passano a pensare che
hanno raggiunto il secondo \emph{jhāna}. Non pensate e non speculate su tutto.
Non ci sono manifesti che annunciano il livello di \emph{samādhi} che state
sperimentando. La realtà è completamente diversa. Non ci sono
segnalazioni come i cartelli stradali che vi indicano ``questa strada va
verso Wat Nong Pah Pong''. Non è così che io vedo la mente. Non fa
proclami.

Sebbene un certo numero di esimi sapienti abbia dato la descrizione del
primo, secondo, terzo e quarto \emph{jhāna}, ciò che è scritto non è altro che
pura informazione esteriore. Se veramente la mente entra in questi
profondi stati di pace, non sa niente di queste descrizioni. Sa, ma ciò
che sa non ha niente a che fare con la teoria che studiamo. Se i dotti
si tengono stretti alle loro teorie e le trasferiscono nella loro
meditazione sedendo e pensando: ``Hmm\ldots{} che sarà questo? E' già il
primo \emph{jhāna}?'' ecco, la pace è finita e non sperimenteranno più niente
che abbia un vero valore. E perché? Perché c'è desiderio, e una volta
che c'è attaccamento cosa succede? La mente esce immediatamente dallo
stato meditativo. Perciò è importante che abbandoniamo completamente
ogni forma di pensiero e di speculazione. Bisogna lasciarli
completamente andare. Considerate solo il corpo, la parola e la mente e
scavate a fondo nella pratica. Osservate il lavorio della mente, ma non
trascinatevi dentro anche i libri di Dhamma altrimenti diventa tutto una
gran confusione, perché niente in quei libri corrisponde esattamente
alla realtà delle cose così come sono.

La gente che studia molto, che è piena di conoscenze teoriche,
generalmente non riesce bene nella pratica di Dhamma. Si impantana al
livello della pura informazione. La verità è che il cuore e la mente non
possono essere misurati su parametri esterni. Se la mente diventa
tranquilla, lasciatela semplicemente essere tranquilla. Esistono dei
livelli di pace molto profondi.

Personalmente io non ne so molto di
teoria. Ero monaco già da tre anni, ma continuavo a chiedermi cosa fosse
veramente il \emph{samādhi}. Cercavo di pensarci e di raffigurarmelo durante la
meditazione, ma la mente diventava sempre più agitata e distratta,
ancora più di quanto lo fosse prima. Anzi, aumentò pure il numero dei
pensieri. Quando non meditavo stavo più calmo. Ragazzi, era veramente
difficile, esasperante! Ma, sebbene incontrassi tanti ostacoli, non
gettai mai la spugna. Semplicemente continuai. Quando non cercavo di
fare qualcosa in particolare, la mente era relativamente a suo agio. Ma
ogni volta che decidevo di concentrare la mente in \emph{samādhi}, ne perdevo
completamente il controllo. Mi chiedevo: ``Ma che capita? Perché succede
questo?''.

Più tardi cominciai a capire che la meditazione è paragonabile al
processo del respiro. Se decidiamo di intervenire sul respiro,
rendendolo leggero, profondo o solo `giusto', vedremo che è difficile
perfino respirare. Tuttavia se ce ne andiamo a fare una passeggiata e
neppure siamo consapevoli dell'inspirazione ed espirazione, è una cosa
rilassante. Perciò riflettei: ``Ah, Forse è così che funziona! Quando,
durante il giorno, ci si muove nelle faccende normali senza concentrare
l'attenzione sul respiro, il respiro causa sofferenza? No, ci si sente
semplicemente rilassati''. Ma se io mi sedevo e prendevo la forte
determinazione di rendere la mente tranquilla, contemporaneamente davo
spazio all'attaccamento e all'avidità. Quando cercavo di controllare il
respiro per farlo diventare leggero o profondo, semplicemente mettevo in
moto più stress di quanto ne avessi prima. Perché? Perché la forza di
volontà che stavo usando era macchiata di attaccamento e avidità. Non
sapevo ciò che stava capitando. Tutta quella frustrazione e fatica erano
causate dal fatto che portavo nella meditazione l'attaccamento.

\Section{Una pace incrollabile}

Una volta stavo in un monastero della foresta distante poco meno di un
chilometro da un villaggio. Una sera, sul tardi, i paesani festeggiavano
e stavano facendo molto chiasso, mentre io praticavo la meditazione
camminata. Dovevano essere circa le 11 di sera e mi sentivo un po'
strano. Già da mezzogiorno avevo questa strana sensazione. La mente era
quieta, quasi senza pensieri. Mi sentivo rilassato e completamente a mio
agio. Continuai a fare meditazione camminata fino a che mi sentii
stanco; allora andai a sedermi nella capanna con il tetto di paglia. Mi
stavo sedendo, quando in modo sorprendente, senza neppure avere il tempo
di incrociare le gambe, la mia mente sentì il bisogno di immergersi in
uno stato di profonda pace. Tutto avvenne per conto suo. Appena seduto,
la mente divenne completamente tranquilla. Era solida come una roccia.
Non è che non sentissi i canti e la musica dei paesani, potevo sentirli,
ma potevo anche completamente chiudere fuori la percezione del suono.

Era strano. Se non facevo attenzione al suono, c'era una pace perfetta,
non sentivo niente. Ma se volevo sentire, lo potevo fare e non mi
disturbava affatto. Era come se nella mia mente ci fossero due oggetti
vicini, che però non si toccavano. Potevo constatare che la mente e il
suo oggetto di consapevolezza erano separati e distinti, proprio come
questa sputacchiera e quel bricco dell'acqua. Poi compresi: quando la
mente è in \emph{samādhi}, se dirigete l'attenzione verso l'esterno potete
udire i suoni, ma se la lasciate dimorare nella sua vacuità, allora è
perfettamente silenziosa. Quando il suono veniva percepito potevo vedere
che il conoscere e il suono erano completamente diversi. Riflettei: ``Se
non fosse così com'è, in che altro modo potrebbe essere?'' È così che
era. Le due cose erano completamente separate. Continuai a indagare in
questo modo fino a che la mia comprensione si approfondì ancora di più:
``Ah, questo è importante. Quando si interrompe l'apparente continuità
dei fenomeni, vi è solo pace''. La precedente illusione di continuità
(\emph{santati}) si era trasformata in pace mentale (\emph{santi}). Così
continuai a meditare, mettendoci un grande sforzo. In quel momento la
mente era concentrata solo sulla meditazione, indifferente a tutto il
resto. Se a quel punto avessi smesso di meditare, sarebbe stato solo
perché l'avevo completata. Potevo prendermela con calma, ma non per
pigrizia, stanchezza o noia. Niente affatto. Tutto ciò era assente dal
cuore. C'era solo un equilibrio interiore perfetto, proprio quello
giusto.

Infine feci una pausa, ma fu solo la postura esterna che cambiò. Il
cuore rimase fermo, immobile, infaticabile. Avevo l'intenzione di
riposare, per cui presi un cuscino. Mentre mi piegavo, la mente rimase
calma come prima. Poi, proprio mentre la testa toccava il cuscino, la
consapevolezza della mente cominciò a fluire verso l'interno; non sapevo
dove andasse, continuava a scorrere sempre più in profondità. Era come
una corrente elettrica che passava attraverso un cavo fino
all'interruttore. Quando raggiunse l'interruttore, il corpo esplose con
un boato assordante. Durante tutto ciò la conoscenza era perfettamente
lucida e sottile. Una volta passato quel punto, la mente fu libera di
penetrare profondamente dentro. Arrivò fino a un punto in cui non c'era
assolutamente niente. Nessuna cosa del mondo esteriore avrebbe potuto
penetrare qui. Niente avrebbe potuto raggiungerlo. Rimasi un po' così
dentro, poi la mente si ritirò, per scorrere di nuovo fuori. Quando però
dico che si ritirò non intendo dire che la feci defluire. Io ero
soltanto un osservatore, un testimone, che conosceva. La mente uscì
sempre di più fino a che ritornò ``normale''.

Appena ripresi il mio solito stato di coscienza, mi domandai: ``Cosa è
successo?!''. Subito giunse la risposta, ``Queste cose avvengono per i
fatti loro. Non devi cercare alcuna spiegazione''. La mente rimase
soddisfatta di questa risposta.

Dopo un po' la mente ricominciò a fluire di nuovo verso l'interno. Non
facevo nessuno sforzo cosciente per dirigere la mente. Fece tutto da
sola. Mentre si muoveva sempre più profondamente all'interno, ad un
tratto toccò di nuovo quell'interruttore. Questa volta il mio corpo
esplose in un'infinità di minuscole particelle. Di nuovo la mente fu
libera di penetrare profondamente dentro se stessa. Silenzio assoluto.
Era andata ancora più in profondità di prima. Assolutamente nessuna cosa
esterna poteva raggiungerla. La mente rimase lì per un po', per il tempo
che volle, e poi si ritirò e rifluì fuori. Seguiva un suo impulso e
tutto avveniva da sé. Io non influenzavo né dirigevo la mente in alcun
modo, non la facevo fluire dentro né la ritraevo fuori. Io conoscevo e
guardavo soltanto.

Di nuovo la mente ritornò al suo normale stato di coscienza, e io non mi
chiesi né pensai a ciò che era accaduto. Mentre meditavo, una volta
ancora la mente si volse verso l'interno. Questa volta l'intero cosmo
esplose e si disintegrò in minutissime particelle. La terra, il suolo,
le montagne, i campi e le foreste -- tutto il mondo -- si disintegrò
nell'elemento spazio. La gente era sparita. Tutto era sparito. Questa
terza volta non rimase assolutamente nulla.

La mente, rivolta all'interno, rimase lì per quanto tempo volle. Non
posso dire che capisco esattamente in che modo vi rimase. E' difficile
descrivere ciò che accadde. Non vi sono termini di paragone a cui
riferirmi. Nessuna similitudine è calzante. Questa volta la mente rimase
all'interno per un tempo molto più lungo che dianzi, e venne fuori da
quello stato dopo un bel po'. Quando dissi che ne uscì, non intendo dire
che la feci uscire io o che stavo controllando ciò che avveniva. La
mente fece tutto da sola. Io ero soltanto un osservatore. Alla fine
ritornò nuovamente al suo stato di coscienza normale. Come descrivere
ciò che avvenne per tre volte? Chi può saperlo? Che termini si
potrebbero usare per descriverlo?

\clearpage

\Section{Il potere di samādhi}

Tutto ciò che vi ho detto finora riguarda la mente, che segue la via
della natura. Non è stata una descrizione teorica della mente o di stati
psicologici. Non ce n'era bisogno. Se c'è fede o fiducia, ci arrivate e
lo fate veramente. Non giocherellate soltanto, anzi mettete in gioco la
vostra stessa vita. E quando la pratica raggiunge lo stadio che ho
appena descritto, il mondo intero è completamente a soqquadro. Dopo, la
comprensione della realtà è completamente diversa. La visione delle cose
è completamente trasformata. Se qualcuno vi vedesse in quel momento,
penserebbe che siete impazzito. Se un'esperienza simile avvenisse a uno
che non sa dominarsi, potrebbe diventare veramente matto, perché niente
è più come prima. La gente sembra diversa da prima; ma solo voi la
vedete così. Tutto, assolutamente tutto cambia. I pensieri sono
trasformati: gli altri la pensano in un modo, voi in un altro. Loro
parlano delle cose in un certo modo, voi in un altro. Loro scendono
lungo un sentiero, voi salite per un'altra via. Non siete più come gli
altri esseri umani. Questo modo di vedere le cose non diminuisce, anzi
persiste e va avanti. Provateci. Se è veramente nel modo in cui l'ho
descritto, non dovete cercare molto lontano. Guardate all'interno del
vostro cuore. Questo cuore è fedele, coraggioso, incrollabile, audace.
Questo è il potere del cuore, la sua fonte di forza e di energia. Il
cuore ha questa forza potenziale. Questo è il potere e la forza di
\emph{samādhi}.

A questo punto è sempre e solo il potere e la purezza che la mente
attinge dal \emph{samādhi}. Questo livello di \emph{samādhi}, è un \emph{samādhi} al suo
culmine. La mente ha raggiunto la vetta del \emph{samādhi}; non è solo una
semplice concentrazione momentanea. Se in quel momento passaste alla
meditazione \emph{vipassana}, la contemplazione sarebbe ininterrotta e
porterebbe a profonde intuizioni. Oppure potreste usare quell'energia
concentrata in altri modi. Da questo punto in poi si possono sviluppare
poteri psichici, compiere miracoli o si può usarla in qualsiasi altro
modo. Gli asceti e gli eremiti hanno usato l'energia di \emph{samādhi} per
rendere santa l'acqua, fare talismani o incantesimi. Sono tutte cose
possibili a questo stadio, e a modo loro possono essere benefiche; ma è
come il beneficio dell'alcool. Lo bevete e vi ubriacate.

Questo livello di \emph{samādhi} è un punto di arrivo. Il Buddha si fermò lì e
si riposò. E' la base per la \emph{vipassana} e la contemplazione. Tuttavia non
c'è bisogno di un \emph{samādhi} così profondo per osservare le condizioni che
ci circondano; perciò continuate diligentemente a contemplare il
processo di causa ed effetto. Per farlo concentriamo la pace e la
chiarezza della mente sull'analisi delle cose visibili, dei suoni, degli
odori, dei sapori, delle sensazioni fisiche, dei pensieri e degli stati
mentali che sperimentiamo. Esaminate gli stati d'animo e le emozioni,
sia positive che negative, sia piacevoli che sgradevoli. Esaminate
tutto. E' come se qualcuno, salito su un albero di manghi ne scuotesse i
frutti facendoli cadere, mentre noi da sotto li raccogliamo. Quelli
marci, non li raccogliamo. Raccogliamo solo quelli sani. Non è
stancante, perché non siamo noi che saliamo sull'albero. Noi ci
limitiamo a raccogliere i frutti stando sotto l'albero.

Capite il significato di questa similitudine? Tutto ciò che viene
sperimentato da una mente pacificata porta ad una comprensione più
vasta. Non si creano più concettualizzazioni e proliferazioni su ciò che
viene sperimentato. Ricchezza, fama, biasimo, lode, felicità e
infelicità vengono da sé. E noi stiamo in pace. Siamo saggi. Anzi è
addirittura divertente. E' divertente rovistare in mezzo a tutto questo
e metterlo in ordine. Ciò che la gente chiama bene, male, cattivo, qui,
lì, felicità, infelicità, tutto va raccolto insieme e usato a nostro
beneficio. Qualcun altro è salito sull'albero di mango e sta scuotendo i
rami per farne cadere i frutti verso di noi. Noi semplicemente ci
divertiamo a coglierli senza paura. E comunque di cosa dovremmo aver
paura? E' qualcun altro che sta in cima all'albero e scuote per noi.
Ricchezza, fama, lode, critiche, felicità, infelicità e tutto il resto
non sono che manghi che cadono a terra, e noi li esaminiamo con cuore
sereno. E allora sapremo quali sono quelli buoni e quelli marci.

\Section{Lavorare in armonia con la natura}

Quando cominciamo a usare la pace e la serenità che abbiamo sviluppato
durante la meditazione per contemplare queste cose, allora sorge la
saggezza. Questo è ciò che chiamo saggezza. Questo è \emph{vipassana}. Non è
qualcosa di inventato e costruito. Se siamo saggi, \emph{vipassana} si
svilupperà naturalmente. Non c'è bisogno di etichettare ciò che accade.
Se c'è solo un piccolo lampo di comprensione intuitiva, la chiamiamo
``piccola \emph{vipassana}''. Quando la visione si chiarisce un po' di più, la
chiamiamo ``media \emph{vipassana}''. Se la conoscenza è completamente in
armonia con la Verità, la chiamiamo la ``\emph{vipassana} ultima''.
Personalmente preferisco usare la parola saggezza (\emph{pañña}) invece
di \emph{vipassana}. Se pensiamo di andarci a sedere ogni tanto e praticare la
``meditazione \emph{vipassana}'', avremo parecchie difficoltà. La comprensione
deve nascere dalla pace e dalla tranquillità. Tutto il processo si
svolgerà naturalmente, da solo. Non possiamo forzarlo.

Il Buddha ci ha insegnato che questo processo matura secondo un suo
ritmo. Quando abbiamo raggiunto questo livello di pratica, lasciamo che
si sviluppi a seconda delle nostre capacità innate, delle attitudini
spirituali e dei meriti che abbiamo accumulato nel passato. Però non
smettiamo mai di applicarci con impegno nella pratica, anche se il
progresso, lento o veloce, è comunque fuori dal nostro controllo. E'
come piantare un albero. L'albero sa a che velocità deve crescere. Se
vogliamo che cresca più velocemente di quanto non faccia, è una pura
illusione. Se vogliamo che cresca più lentamente, riconosciamo che anche
questa è un'illusione. Una volta fatto il lavoro, il risultato verrà da
sé, proprio come quando si pianta un albero. Per esempio, mettiamo che
vogliamo piantare una pianticella di peperoncino. Il nostro compito è di
scavare un buco, piantare il virgulto, annaffiarlo, concimarlo e
proteggerlo dagli insetti. Questo è il nostro lavoro, la parte che
dobbiamo fare noi. E' a questo punto che interviene la fede. Che la
pianta di peperoncino cresca o no non dipende da noi. Non è affar
nostro. Non possiamo tirare la pianta, strattonarla in modo che cresca
più in fretta. Non è così che lavora la natura. Il nostro compito è di
innaffiarla e concimarla. Praticare il Dhamma in questo modo rende
pacifici i nostri cuori.

Se realizziamo l'Illuminazione durante questa vita, bene. Se dobbiamo
attendere la prossima, non fa niente. Abbiamo fede e un'incrollabile
fiducia nel Dhamma. Il fatto di progredire velocemente o lentamente
dipende dalle nostre capacità innate, dalle attitudini spirituali e dai
meriti accumulati. Praticare così rende tranquillo il cuore. E' come se
guidassimo un carro a cavalli. Non mettiamo il carro davanti ai cavalli.
O è come arare una risaia, stando davanti e non dietro al bufalo che
tira l'aratro. Ciò che voglio dire è che la mente si proietta oltre se
stessa. Diventa impaziente di avere risultati veloci. Non è questo il
modo di fare. Non camminate davanti al bufalo. Dovete
camminare \emph{dietro} al bufalo.

E' come quella pianta di peperoncino che facciamo crescere. Datele acqua
e concime e sarà lei stessa a fare il lavoro di assorbirli. Quando le
termiti e le formiche vengono a infestarla, le cacciamo. Basta questo
affinché la pianta cresca bella con le sue proprie forze, e una volta
che cresce bene, non forzatela a produrre fiori perché riteniamo che sia
il tempo della fioritura. Non è affar nostro. Creerà solo inutili
disagi. Lasciatela fiorire a tempo debito. E appena i fiori sbocciano,
non aspettatevi che subito portino frutti. Non basatevi sulla
coercizione. E' una causa di sofferenza! Quando capiamo tutto questo,
capiamo anche quali sono, o non sono, le nostre responsabilità. Ognuno
ha il proprio compito da adempiere. La mente sa quale è il suo ruolo nel
lavoro che va fatto. Se la mente non capisce il suo ruolo, cercherà di
forzare la pianticella a produrre peperoncini lo stesso giorno che la
piantiamo. La mente insisterà perché cresca, fiorisca e produca i frutti
tutto in un sol giorno.

Questa non è altro che la Seconda Nobile Verità: l'attaccamento causa la
sofferenza. Se siamo consci di questa Verità e la contempliamo, capiremo
che cercare di forzare i risultati della nostra pratica di Dhamma è una
pura illusione. E' sbagliato. Capendone il funzionamento, saremo in
grado di lasciar andare e di permettere alle cose di maturare a seconda
delle nostre capacità innate, delle attitudini spirituali che possediamo
e dei meriti che abbiamo accumulato. Noi continuiamo a fare la nostra
parte. Non preoccupatevi che ci voglia troppo tempo. Anche se ci
volessero centinaia o migliaia di vite per realizzare l'Illuminazione, e
allora? Per quante vite ci vorranno, noi continueremo a praticare con
cuore sereno, a nostro agio, al nostro ritmo. Una volta che la mente è
``entrata nella corrente'', non c'è più niente da temere. Vuol dire che
non esiste nemmeno la possibilità che venga compiuta la più piccola
azione cattiva. Il Buddha ha detto che la mente di
un \emph{sotapanna} -- uno che ha ottenuto il primo grado di
illuminazione - è entrata nella corrente del Dhamma che fluisce verso
l'illuminazione. Un \emph{sotapanna} non dovrà più sperimentare gli stati più
miseri di esistenza, non cadrà più nell'inferno. E come potrebbe infatti
cadere nell'inferno quando ormai la mente ha abbandonato il male? Ha
visto il pericolo di produrre un \emph{kamma} cattivo. Anche se cercate di
forzarlo a fare o a dire qualcosa di cattivo, ne sarebbe incapace, ed è
per questo che non corre più il pericolo di cadere nell'inferno o negli
stati di esistenza più bassi. La sua mente fluisce nella corrente del
Dhamma.

Una volta che siete nella corrente, sapete quali sono le vostre
responsabilità. Capite che lavoro va fatto. Sapete come praticare il
Dhamma. Sapete quando metterci sforzo e quando rilassarvi. Comprendete
la mente e il corpo, i processi fisici e mentali, e rinunciate alle cose
che vanno lasciate andare, abbandonandole in continuazione senza la
minima ombra di dubbio.

\Section{Cambiare la propria visione}

Nella mia pratica di Dhamma non ho mai tentato di padroneggiare una
vasta gamma di cose. Anzi, ho puntato solo ad una. Ho raffinato questo
cuore. Mettiamo che stiamo osservando un corpo. Se troviamo che siamo
attratti da un corpo, allora analizziamolo. Osservatelo bene: capelli,
peli, unghie, denti e pelle.%
\pagenote{\emph{Kesā, lomā, nakhā, dantā, taco}: la contemplazione di queste cinque parti del corpo costituisce la prima tecnica meditativa assegnata dal maestro al nuovo monaco o monaca.}
Il Buddha ci ha
insegnato a contemplare accuratamente e ripetutamente queste parti del
corpo. Visualizzatele separatamente, dividetele, toglietene la pelle e
inceneritele. Questo va fatto. Rimanete con questa meditazione fino a
che si consolida fermamente, senza alcuna indecisione. Guardate tutti
allo stesso modo. Per esempio, quando la mattina i monaci e i novizi
vanno al villaggio per l'elemosina, chiunque vedano, sia un altro monaco
o un altra persona, cerchino di immaginare lui o lei come un corpo
morto, un cadavere ambulante che cammina sulla strada davanti a loro.
Rimanete concentrati su questa percezione. E' così che si incrementa lo
sforzo. E questo porta alla maturità e allo sviluppo. Quando vedete una
giovane donna che vi attrae, immaginatela come un cadavere ambulante,
con il corpo putrefatto, esalante puzza di decomposizione. Vedete tutti
sotto questa luce. E non fateli avvicinare troppo! Non permettete
all'infatuazione di prendere piede nel vostro cuore. Se li percepite
putridi e puzzolenti, vi assicuro che l'infatuazione non continuerà.
Contemplate fino ad essere sicuri di quello che vedete, fino a che la
visione non sia chiara, fino a che non ne diventiate esperti. Per
qualsiasi via poi vi incamminiate, non andrete più fuori strada.
Metteteci tutto il cuore. Ogni volta che vedete qualcuno, sarà come
vedere un cadavere. Sia maschio che femmina, guardatene il corpo come se
fosse morto. E non dimenticate di vedere il vostro come morto. In fondo
è tutto quello che rimarrà di essi. Cercate di sviluppare questo modo di
vedere il più completamente possibile. Esercitatevi finché diventa
sempre più parte della vostra mente. Vi assicuro che, al lato pratico, è
un gran divertimento. Ma se vi affannate a leggerlo nei libri,
incontrerete serie difficoltà. Dovete farlo. E fatelo con assoluta
sincerità. Fatelo fino a che questa meditazione non diventa parte di
voi. Fate della realizzazione della Verità il vostro scopo. Se siete
motivati dal desiderio di trascendere la sofferenza, allora sarete sul
sentiero giusto.

In questi tempi ci sono molti insegnanti di \emph{vipassana} e una vasta gamma
di tecniche. Vi dirò solo: fare \emph{vipassana} non è facile. Non possiamo
semplicemente saltarci dentro. Non funzionerà se non si parte da un alto
livello di moralità. Scopritelo voi stessi. La disciplina morale e i
precetti sono necessari, perché se il nostro comportamento, le nostre
azioni e la nostra parola non sono impeccabili, non riusciremo mai a
star ritti sulle nostre due gambe. La meditazione senza moralità è come
cercare di evitare una parte importante del Sentiero. Allo stesso modo,
certe volte sentiamo dire: ``Non c'è bisogno di sviluppare la
tranquillità; lasciala perdere e passa direttamente alla meditazione
\emph{vipassana}''. A dire cose di questo tipo sono quegli individui
superficiali, che cercano sempre scappatoie. Dicono che non bisogna
preoccuparsi della disciplina morale. Non è un giochetto sostenere e
raffinare la propria virtù, anzi è una sfida. Se potessimo tralasciare
tutti gli insegnamenti sul comportamento morale, sarebbe tutto più
facile, vero? Ogni volta che incontriamo una difficoltà non faremmo
altro che evitarla, saltandola a piè pari. Naturalmente tutti vorremmo
poter evitare le difficoltà.

Una volta incontrai un monaco che mi disse che lui era un vero
meditante. Mi chiese il permesso di stare con noi e si informò sul
programma e sul livello di disciplina monastica. Gli spiegai che in
questo monastero vivevamo secondo il \emph{Vinaya}, il codice di
disciplina monastica del Buddha, e che se voleva venire a stare qui
doveva rinunciare al suo denaro personale e al rifornimento personale di
cibo. Mi disse che la sua pratica era ``non attaccamento a tutte le
convenzioni''. Gli risposi che non capivo di cosa stesse parlando. ``E
se io stessi qui, tenendo il denaro ma senza attaccarmi ad esso? Il
denaro è solo una convenzione.'' Gli dissi che certo, non c'era
problema. ``Se puoi mangiare il sale e non trovarlo salato, allora puoi
usare il denaro senza attaccarti ad esso''. Stava dicendo cose senza
senso. In effetti era troppo pigro per seguire tutti i dettagli
del \emph{Vinaya}. Ve lo ripeto, è difficile. ``Quando puoi mangiare il
sale e onestamente assicurarmi che non è salato, allora ti prenderò sul
serio. E se mi dici che non è salato allora te ne darò un sacco intero
da mangiare. Provaci soltanto. Veramente non avrà il gusto di sale? Il
non attaccamento alle convenzioni non è soltanto questione di essere
abili con le parole. Se parli così, non puoi stare con me''. Allora se
ne andò.

Dobbiamo cercare di praticare e mantenere la virtù. I monaci devono
esercitarsi con le pratiche ascetiche,%
\pagenote{\emph{Dhutaṇga}: pratiche ascetiche raccomandate dal Buddha come un ``mezzo per scuotersi di dosso le contaminazioni''. Comprendono 13 strette osservanze che aiutano a coltivare il senso di accontentarsi, di rinuncia e di sforzo energico.}
mentre la
gente a casa dovrebbe praticare i cinque precetti.%
\pagenote{\emph{Cinque precetti}: le cinque linee guida basilari per esercitarsi a compiere solo azioni virtuose del corpo e della parola: astenersi dall'uccidere, astenersi dal rubare, avere una condotta sessuale responsabile, astenersi dal mentire, seminare discordia e dalla parola dura o frivola, astenersi dall'assumere intossicanti.}
Bisogna tentare di essere impeccabili in tutto ciò che si dice e si fa.
Bisogna coltivare la bontà con tutte le nostre forze, e continuare a
rinforzarla man mano.

Quando cominciate a coltivare la serenità di \emph{samatha}, non commettete
l'errore di provarci una volta o due e poi di rinunciarci perché trovate
che la mente non si tranquillizza. Non è il modo giusto di fare.
Dobbiamo coltivare la meditazione per un lungo periodo di tempo. Perché
dobbiamo metterci tanto? Provate a pensarci. Per quanti anni abbiamo
permesso alla mente di vagare dappertutto? Per quanti anni non abbiamo
praticato la meditazione \emph{samatha}? Ogni volta che la mente ci imponeva di
seguirla su una certa via, noi ci precipitavamo dietro di essa. Per
calmare questa mente vagante, per fermarla, per pacificarla, non
basteranno un paio di mesi di meditazione. Considerate questo punto.

Quando cominciate ad esercitare la mente affinché sia in pace in ogni
situazione, dovete capire che all'inizio, quando sorge un'emozione
inquinante, la mente non sarà affatto in pace. Sarà distratta e fuori
controllo. Perché? Perché c'è attaccamento. Non vogliamo che la mente
pensi. Non vogliamo sperimentare nessuna disattenzione o emozione. Non
volere equivale ad attaccamento, attaccamento per la non esistenza. Più
vogliamo non sperimentare certe cose, più le invitiamo e le facciamo
entrare in noi. ``Non voglio queste cose e allora perché continuano a
venire da me? Non voglio che vada in questo modo e allora perché va in
questo modo?''. Eccoci al punto! Vogliamo che le cose vadano in un certo
modo, perché non capiamo la nostra stessa mente. Non ci vuole
un'eternità per capire che baloccarsi con queste cose è un errore.
Infine quando consideriamo bene la cosa, ci arriviamo: ``Oh, queste cose
vengono perché sono io a farle venire!''.

Volere non sperimentare qualcosa, volere stare in pace, volere non
essere distratti o agitati, tutto ciò è attaccamento. E' una palla di
ferro incandescente. Ma non fa niente. Continuate con la pratica. Ogni
volta che sperimentate uno stato d'animo o un'emozione esaminateli nei
termini delle loro qualità di impermanenza, insoddisfazione e non sé e
cacciateli in una di queste tre categorie. Poi riflettete e indagate:
queste emozioni inquinanti sono quasi sempre accompagnate da
un'eccessiva quantità di pensieri. Quando ci lasciamo guidare da uno
stato d'animo, la proliferazione mentale gli tiene dietro. Il pensiero e
la saggezza sono due cose diametralmente opposte. Il pensiero non fa che
reagire allo stato d'animo e a seguirlo, e vanno avanti così senza una
fine in vista. Ma se la saggezza è all'opera, essa fermerà la mente. La
mente si ferma e non va più in giro. C'è solo la conoscenza e il
riconoscere ciò che si è appena sperimentato: quando sorge questa
emozione, la mente è così; quando questo stato d'animo sorge, è in
quest'altra maniera. Incrementiamo solo il ``conoscere''. Alla fine
realizziamo: ``Ehi, tutto questo pensare, tutto questo inutile
chiacchiericcio, questo preoccuparsi e giudicare, è tutto insensato e
immaginario. Tutto è impermanente, insoddisfacente e non-me o mio''.
Cacciatelo in una di queste tre caratteristiche onnicomprensive e
acquietate l'agitazione. In tal modo lo tagliate alla radice. Più tardi,
seduti in meditazione, si rifarà sentire. Tenetelo d'occhio, spiatelo.

E' come allevare bufali d'acqua. Ci sono: il contadino, qualche pianta
di riso e il bufalo. Naturalmente il bufalo vuole mangiare quelle piante
di riso. Ai bufali piace mangiare le piante di riso, vero? La vostra
mente è un bufalo. Le emozioni inquinanti sono come le pianticelle di
riso. Il conoscere è il contadino. La pratica del Dhamma è proprio così.
Non diversa. Fate voi stessi il paragone. Quando sorvegliate un bufalo,
cosa fate? Lo liberate, gli permettete di andarsene in giro libero, ma
contemporaneamente lo tenete d'occhio. Se si avvicina troppo alle piante
di riso, lo richiamate. Quando il bufalo vi sente, si allontana da esse.
Ma non siate distratti, non dimenticatevi del bufalo. Se avete un bufalo
ostinato che non fa attenzione ai vostri richiami, prendete un bastone e
dategli una forte randellata sul dorso. Vedrete che non oserà
avvicinarsi più alle pianticelle di riso. Ma non lasciatevi andare a
fare una siesta. Se vi sdraiate e sonnecchiate, quelle piante di riso
faranno parte del passato. La pratica del Dhamma è la stessa cosa;
controllate la mente; il conoscere stesso fa da sorvegliante alla mente.

``Quelli che sorvegliano accuratamente la loro mente saranno liberati
dalle trappole di \emph{Māra}''.%
\pagenote{\emph{Māra}: la personificazione buddhista delle forze antagoniste all'Illuminazione.}
Eppure anche
questa mente che conosce è sempre la mente, e allora chi osserva la
mente? Una tale idea vi può procurare parecchia confusione. La mente è
una cosa, il conoscere è un'altra; eppure il conoscere trae origine da
quella stessa mente. Che vuol dire conoscere la mente? Com'è imbattersi
negli stati d'animo e nelle emozioni? Com'è stare senza alcuna emozione
inquinante? Ciò che sa cosa sono queste cose è ciò che intendiamo per
``conoscere''. Il conoscere segue attentamente la mente, ed è da questo
conoscere che nasce la saggezza. La mente è ciò che pensa e rimane
impigliata nelle emozioni, una dopo l'altra, proprio come il nostro
bufalo. Qualsiasi direzione essa prenda, state all'erta. Come potrebbe
sfuggirvi? Se gironzola intorno alle piante di riso, urlatele dietro. Se
non ascolta, prendete un bastone e giù una randellata! E' così che
frustrate l'attaccamento.

Addestrare la mente non è quindi diverso. Quando la mente sperimenta
un'emozione e subito vi si aggrappa, è compito del conoscere fare da
insegnante. Esamina lo stato d'animo per vedere se è positivo o
negativo. Spiega alla mente come funziona la legge di causa ed effetto.
E quando essa si aggrappa di nuovo a qualcosa che ritiene gradevole, il
conoscere deve di nuovo insegnare alla mente, di nuovo deve spiegare la
legge di causa ed effetto, fino a che la mente riesce a mettere tutto da
parte. Questo porta alla pace della mente. Quando infine scopre che
tutto ciò che afferra e a cui si aggrappa è di per sé indesiderabile,
semplicemente la smette. Non le interessano più quelle cose, perché
incontra uno sbarramento di rimproveri e rabbuffi. Opponetevi alla
bramosia della mente con determinazione. Sfidatela apertamente, fino a
che l'insegnamento penetrerà nel cuore. E' così che addestrate la mente.

Fin dal tempo in cui mi sono ritirato nella foresta a meditare, ho
praticato in questo modo. Quando addestro i miei discepoli, lo faccio in
questa stessa maniera. Perché voglio che vedano la verità, piuttosto che
leggere cosa dicono le scritture; voglio che vedano se i loro cuori sono
liberi dal pensiero concettuale. Quando avviene la liberazione, ne siete
consapevoli; e quando non c'è ancora la liberazione, contemplate il
processo per cui una cosa è causa di un'altra. Contemplate fino a che
sapete e conoscete tutto ciò ripetutamente e completamente. Una volta
che l'avete penetrato attraverso una conoscenza diretta, se ne andrà per
i fatti suoi. Quando interviene qualcos'altro e vi sentite impantanati,
allora indagate. Non smettete finché quello non ha lasciato la presa.
Continuate a indagare proprio in quel punto. Personalmente, è così che
mi sono esercitato, perché il Buddha ha detto che dovete arrivare alla
conoscenza da soli. Tutti i saggi conoscono la verità da soli. Dovete
anche voi scoprirla nelle profondità del vostro cuore. Conoscete voi
stessi.

Se avete fiducia in ciò che conoscete e se vi fidate di voi, vi sentite
a vostro agio anche se altri vi criticano o vi lodano. Siete a vostro
agio qualsiasi cosa dicano gli altri. Perché? Perché conoscete voi
stessi. Se uno vi riempie di lodi, ma voi non ne siete pienamente
meritevoli, ci credete veramente a quello che vi dicono? Naturalmente
no. Andate avanti semplicemente con la vostra pratica del Dhamma. Quando
uno che non ha fiducia in ciò che sa viene lodato, tende a crederci e
questo distorce la sua percezione. Allo stesso modo, quando qualcuno vi
critica, fatevi un bell'esame di coscienza e ditevi: ``No, ciò che hanno
detto non è vero. Mi accusano di aver sbagliato, ma non è così. Le loro
accuse non sono valide''. Se la situazione è questa, a che pro'
arrabbiarsi con loro? Le loro parole non sono sincere. Però se siete in
errore proprio come loro vi accusano, allora la critica è corretta. Se
così è, a che pro' arrabbiarsi con loro? Quando riuscite a pensare in
questa maniera, vedrete che la vita è veramente pacifica e confortevole.
Niente di ciò che avviene è sbagliato. Tutto è Dhamma. E' così che io ho
praticato.

\Section{Seguire la via di mezzo}

E' il sentiero più breve e più diretto. Se veniste a discutere con me
alcuni punti del Dhamma, io non prenderei parte alla discussione. Invece
di confutare, vi offrirei alcune riflessioni da tenere presenti. Cercate
di capire ciò che il Buddha ha insegnato: lasciate andare tutto.
Lasciate andare con conoscenza e consapevolezza. Senza conoscenza e
consapevolezza, il lasciar andare non è molto diverso da quello dei buoi
e dei bufali. Se non ci mettete il cuore, il lasciar andare non è quello
giusto. Lasciate andare perché avete capito la realtà convenzionale.
Questo è non-attaccamento. Il Buddha ha insegnato che negli stadi
iniziali della pratica del Dhamma dovete lavorare molto, sviluppare
completamente le cose e attaccarvi molto. Attaccarvi al Buddha.
Attaccarvi al Dhamma. Attaccarvi al Sangha. Attaccarvi con fermezza,
profondamente. Questo è ciò che il Buddha ha insegnato. Attaccarvi con
sincerità e perseveranza e mantenere la presa.

Durante la mia ricerca, ho provato tutti i metodi possibili di
contemplazione. Ho sacrificato la mia vita al Dhamma perché avevo fede
nella realtà dell'Illuminazione e del Sentiero che vi conduce. Queste
cose esistono veramente, proprio come ha detto il Buddha. Ma per
realizzarle è necessario praticare, praticare rettamente. Bisogna
spingersi fino al massimo delle proprie possibilità. Ci vuole il
coraggio di esercitarsi, di riflettere e di cambiare radicalmente. Ci
vuole il coraggio di fare veramente tutto ciò che è necessario. E come
lo fate? Addestrando il cuore. I pensieri in testa ci dicono di andare
in una certa direzione, ma il Buddha ci dice di andare in un'altra.
Perché è necessario addestrarci? Perché il cuore è completamente
ricoperto da incrostazioni inquinanti. Un cuore non ancora trasformato
dall'esercizio è così. E' inaffidabile, per cui non credeteci. Non è
ancora virtuoso. Come possiamo avere fiducia in un cuore che non ha
purezza e chiarezza? Perciò il Buddha ci mise in guardia dal confidare
in un cuore impuro. Inizialmente il cuore è solo al servizio delle
contaminazioni, e quando i due stanno a lungo assieme, il cuore si
guasta e si corrompe. Per questo il Buddha ci ha detto di non riporre
fiducia nel cuore.

Se consideriamo attentamente la nostra disciplina monastica, vedremo che
tutto si riduce ad esercitare il cuore. E ogni volta che addestriamo il
cuore ci sentiamo agitati e infastiditi. Non appena proviamo agitazione
o fastidio, cominciamo a lamentarci ``Ragazzi, questa pratica è
veramente difficile! E' quasi impossibile''. Ma il Buddha non la pensava
così. Egli pensava che quando l'addestramento ci procura agitazione e
disagio, vuol dire che siamo sulla strada giusta. Ma noi non la pensiamo
così. Pensiamo che siano segni di qualcosa di sbagliato. Questo
malinteso fa sembrare la pratica molto difficile. All'inizio sentiamo
agitazione, siamo nervosi e allora pensiamo di essere fuori strada.
Tutti vogliono star bene, ma non si chiedono se sia corretto o meno.
Quando andiamo contro le contaminazioni e sfidiamo la nostra bramosia, è
normale che soffriamo. Ci sentiamo agitati, sconvolti, a disagio e
infine lasciamo perdere. Pensiamo di essere sulla strada sbagliata.

Il Buddha invece avrebbe detto che siamo su quella giusta. Stiamo
affrontando le nostre impurità e sono loro che ci procurano agitazione e
disagio. Ma pensiamo invece di essere noi stessi agitati e a disagio. Il
Buddha invece ci ha detto che sono le impurità che saltano su e si
agitano. E' la stessa cosa per tutti.

Per questo la pratica del Dhamma è così impegnativa. Le persone non
esaminano le cose con chiarezza. Generalmente perdono il Sentiero,
andando o nella direzione dell'auto-indulgenza o dell'auto-punizione. Si
bloccano su uno di questi due estremi. Da una parte ci sono quelli a cui
piace indulgere in tutto ciò che il cuore desidera. Fanno tutto ciò che
si sentono di fare. Gli piace sedere comodi. Gli piace sdraiarsi e
stirarsi comodamente. Tutto quello che fanno, ha lo scopo di farli stare
comodi e a loro agio. Questo è ciò che considero auto-indulgenza:
attaccarsi alla sensazione confortevole. Con un tale atteggiamento come
può progredire la pratica del Dhamma?

Se non riusciamo più a indulgere in comodità, sensualità e benessere, ci
irritiamo. Ci sentiamo defraudati, ci arrabbiamo e perciò ne soffriamo a
causa di questi sentimenti. Questo è uscire dal Sentiero in direzione
dell'auto-punizione. Questa non è la via del saggio, né la via di chi è
calmo. Il Buddha ci mise in guardia dal cadere in uno di questi due
estremi: dell'auto-indulgenza o dell'auto-punizione. Quando sperimentate
un piacere, siatene consci con consapevolezza. Quando sperimentate
rabbia, malevolenza e irritazione, rendetevi conto che non state
seguendo le orme del Buddha. Non è la via per chi cerca la pace, ma la
strada della gente comune. Un monaco che cerca la pace non percorre
queste vie. Procede diritto nel mezzo, lasciando l'auto-indulgenza a
sinistra e l'auto-punizione a destra. Questa è la corretta pratica del
Dhamma.

Se intraprendete questa pratica monastica, dovete camminare sulla Via di
Mezzo, senza lasciarvi dominare dalla felicità o dall'infelicità.
Lasciatele perdere. Invece sembra proprio che ci spingano da una parte
all'altra. Sembriamo il battaglio di una campana, spinto avanti e
indietro da un lato all'altro. Nella Via di Mezzo si lascia andare sia
la felicità che l'infelicità; la giusta pratica è quella che sta nel
mezzo. Quando ci colpisce il desiderio di felicità e non riusciamo a
soddisfarlo, proviamo dolore.

Camminare lungo la Via di Mezzo del Buddha è impegnativo e arduo. Ci
sono solo quei due estremi, il buono e il cattivo. Se crediamo in quello
che essi ci dicono, dobbiamo seguire i loro ordini. Se ci arrabbiamo con
qualcuno, immediatamente andiamo a cercare un bastone per picchiarlo.
Niente tolleranza e pazienza. Se amiamo qualcuno vogliamo accarezzarlo
dalla testa ai piedi. Ho ragione? Questi due estremi non considerano
affatto la parte in mezzo. Non è ciò che il Buddha ci ha raccomandato.
Egli ci ha insegnato a lasciar gradualmente perdere tutto ciò. La sua
pratica segue un sentiero che porta fuori dall'esistenza, lontano dalla
rinascita -- un sentiero libero dal divenire, dalla nascita, dalla
felicità, dall'infelicità, dal bene e dal male.

La gente che brama l'esistenza non vede ciò che sta nel mezzo. Escono
dal Sentiero puntando verso la felicità e poi, ignorando completamente
il mezzo, passano all'altro lato, all'insoddisfazione e all'irritazione.
Non fanno altro che evitare il centro. Questo punto sacro è per loro
invisibile, mentre passano correndo da una parte all'altra. Non si
fermano lì dove non c'è né esistenza né rinascita. Non gli piace, per
cui non si soffermano. O scendono da casa e vengono morsi da un cane o
volano in alto e sono beccati da un avvoltoio. Questa è l'esistenza.

L'umanità è cieca a ciò che non ha esistenza, che non ha rinascita. Il
cuore umano è cieco a questo, perciò non fa che passargli accanto o
evitarlo. La Via di Mezzo del Buddha, il Sentiero della retta pratica
del Dhamma, trascende l'esistenza e la rinascita. La mente che è al di
là sia della purezza che dell'impurità è libera. Questo è il sentiero
del saggio che sta in pace. Se non lo percorriamo non saremo mai dei
saggi pacifici. Questa pace non avrà mai occasione di fiorire. Perché? A
causa dell'esistenza e della rinascita. Perché c'è nascita e morte. Il
sentiero del Buddha non ha né nascita né morte. Non ha alti e bassi. Non
ha felicità o sofferenza. Non ha bene o male. E' un sentiero diretto. E'
il sentiero della quiete e della calma. E' pacifico, libero dal piacere
e dal dolore, dalla felicità e dall'angoscia. Questo è il modo di
praticare il Dhamma. Quando si sperimenta ciò, la mente può fermarsi.
Può smettere di porre domande. Non c'è più bisogno di andare in cerca di
risposte. Ecco perché il Buddha disse che il Dhamma è qualcosa che il
saggio conosce da solo, direttamente. Non c'è bisogno di chiederlo a
nessuno. Capiamo perfettamente da noi stessi, senza ombra di dubbio, che
le cose sono esattamente come il Buddha ha detto che sono.

\Section{Dedizione alla pratica}

Vi ho raccontato alcuni episodi della mia pratica. Io non ho grandi
conoscenze. Non ho studiato molto. Quello che ho studiato è questo mio
cuore e questa mia mente e ho imparato in modo naturale attraverso
l'esperienza, i tentativi e gli sbagli. Quando mi piaceva qualcosa,
esaminavo quello che mi stava accadendo e a che cosa portava quel
desiderio. Inevitabilmente mi spingeva verso una sofferenza futura. La
mia pratica era quella di osservare me stesso. Man mano che la
comprensione e l'intuizione profonda si approfondivano, riuscii a
conoscere me stesso.

Praticate con una irremovibile dedizione! Se volete praticare il Dhamma,
cercate di non pensare troppo. Se durante la meditazione vi accorgete
che vi state sforzando per raggiungere risultati specifici, allora è
meglio che smettiate. Quando la mente si assesta nella pace e cominciate
a pensare: ``Ecco finalmente! Ci sono, vero? E' proprio così'', allora
fermatevi. Prendete tutte le vostre conoscenze analitiche e teoriche,
fatene un fagotto e riponetele in un baule. E non tiratele fuori per
discuterle o insegnarle. Non è questo il tipo di conoscenza che penetra
all'interno. Sono altri tipi di conoscenza.

Quando si vede qualcosa nella realtà, non sempre corrisponde alla
descrizione fattane per iscritto. Per esempio, mettiamo di scrivere la
parola ``desiderio sensuale''. Quando il desiderio sensuale invade
veramente il cuore, è impossibile che la parola scritta trasmetta lo
stesso significato della realtà. Lo stesso accade con la ``rabbia''.
Possiamo scrivere la parola in maiuscole su un cartello, ma quando
veramente siamo arrabbiati l'esperienza non ha niente a che fare con le
parole. Non abbiamo neanche il tempo di leggerle quelle parole, prima
che il cuore sia inghiottito dalla rabbia.

Questo è un punto molto importante. Gli insegnamenti teorici hanno la
loro importanza, ma bisogna che penetrino nel cuore. Devono essere
interiorizzati. Non possiamo conoscere veramente il Dhamma se non lo
interiorizziamo. Non lo vediamo realmente. Anch'io non facevo eccezione.
Non ho una vasta conoscenza, ma ho studiato abbastanza da passare alcuni
esami di teoria buddhista. Un giorno ebbi occasione di ascoltare un
discorso di Dhamma tenuto da un maestro di meditazione. Mentre ascoltavo
cominciai ad avere pensieri poco rispettosi. Non sapevo come ascoltare
un vero discorso di Dhamma. Non riuscivo a capire che cosa stesse
dicendo quel monaco errante. Era come se il suo insegnamento provenisse
da una sua esperienza diretta, come se fosse in contatto con la verità.

Col passare del tempo, man mano che acquisivo una certa padronanza
diretta della pratica, vidi da me la verità di cui parlava quel monaco.
Compresi in che modo comprendere. E così sulla sua scia sorse una
comprensione diretta. Il Dhamma stava mettendo radici nel mio cuore e
nella mia mente. Ci volle molto, molto tempo prima che capissi che tutto
quello che aveva detto quel monaco errante, proveniva da ciò che aveva
visto lui stesso. Il Dhamma che insegnava proveniva direttamente dalla
sua esperienza, non da un libro. Parlava secondo la sua comprensione e
la sua intuizione profonda. Quando io stesso percorsi il Sentiero, feci
l'esperienza di tutti i dettagli che aveva descritto e dovetti ammettere
che aveva ragione. Perciò andai avanti.

Cercate di afferrare ogni occasione per praticare il Dhamma. Che sia un
momento tranquillo o no, non preoccupatevi di questo. La cosa più
importante è mettere in moto la ruota della pratica e creare le cause
per la futura liberazione. Se avete fatto un buon lavoro, non c'è
bisogno di preoccuparsi dei risultati. Non angosciatevi pensando che non
state ottenendo alcun risultato. L'angoscia non è pace. D'altronde se
non fate il lavoro, come potete aspettarvi dei risultati? Come potete
credere di poter vedere? Solo chi cerca può scoprire. Solo chi mangia,
si sazia. Tutto ciò che ci circonda è falso. Continuare a rendersene
conto, anche per decine di volte, è già un bene. Quel tizio continua a
raccontarci sempre le stesse bugie e storielle. Se ci rendiamo conto che
sta mentendo, non è poi così male, ma certe volte ci vuole parecchio
tempo prima di accorgercene. Quel tizio proverà a raggirarci ancora e
ancora.

Praticare il Dhamma vuol dire mantenere la virtù, sviluppare \emph{samādhi} e
coltivare la saggezza nel cuore. Ricordatevi e riflettete sulla Triplice
Gemma: il Buddha, il Dhamma e il Sangha. Abbandonate completamente
tutto, senza eccezioni. Le nostre stesse azioni sono le cause e le
condizioni che matureranno già in questa vita. Perciò impegnatevi
sinceramente nella pratica.

Anche se dobbiamo sederci su una sedia per meditare, possiamo lo stesso
tenere fissa l'attenzione. All'inizio non la terremo su molte cose, solo
sul respiro. Se preferite, potete ripetere mentalmente le parole
``Buddha'', ``Dhamma'' o ``Sangha'' insieme ad ogni respiro. Mentre
tenete fissa l'attenzione cercate di non controllare il respiro. Se il
respiro sembra laborioso o difficile, significa che non abbiamo il
giusto approccio. Fino a che non ci sentiremo a nostro agio con il
respiro, sembrerà sempre o troppo superficiale o troppo profondo, troppo
sottile o troppo grossolano. Ma una volta che ci rilassiamo nel respiro,
trovandolo piacevole e comodo, chiaramente consapevoli di ogni
ispirazione ed espirazione, allora possiamo dire che ne abbiamo compreso
il senso. Se non lo facciamo nel modo corretto, perderemo il respiro. Se
questo dovesse accadere allora è meglio smettere per un po' e rimettere
a fuoco la consapevolezza.

Se durante la meditazione sentite l'impulso di sperimentare fenomeni
psichici o se la mente diventa luminosa e radiante, o se avete visioni
di palazzi celesti, ecc. non abbiate paura. Siate semplicemente
consapevoli di ciò che state sperimentando e continuate a meditare. Ogni
tanto, può accadere che dopo un po' il respiro rallenti fino a
scomparire. Vi sembra di non sentire più il respiro e vi allarmate. Non
preoccupatevi, non c'è niente di cui essere spaventati. Che il respiro
sia cessato, lo pensate soltanto; in effetti il respiro è sempre lì, ma
lavora a un livello molto più sottile del solito. Dopo un po' il respiro
tornerà normale da solo.

All'inizio concentratevi solo per rendere calma e tranquilla la mente.
Dovreste arrivare a un tale livello di meditazione da essere in grado di
entrare volontariamente in uno stato di pace, sia che siate seduto in
poltrona, o che siate in battello o in qualsiasi altro luogo. Quando
salite in treno accomodatevi e portate subito la mente in uno stato di
pace. Ovunque siate potete sempre sistemarvi in qualche modo. Questa
capacità dimostra che vi state familiarizzando con il Sentiero. Poi
cominciate ad indagare. Utilizzate il potere di questa mente tranquilla
per indagare nella vostra esperienza. Alcune volte riguarda ciò che
udite, altre ciò che vedete, odorate, gustate, provate col corpo o
percepite e pensate col cuore o con la mente. Qualsiasi esperienza
sensoriale si presenti, che vi piaccia o no, prendetela come un oggetto
di contemplazione. Siate semplicemente consci di quello che state
sperimentando. Non proiettate significati o interpretazioni sull'oggetto
di consapevolezza sensoriale. Se è buono, sapete solo che è buono; se è
cattivo, sapete solo che è cattivo. Questa è una realtà convenzionale.
Buono o cattivo, è tutto comunque impermanente, insoddisfacente e
non-sé. Tutto è inaffidabile. Non c'è niente per cui vale la pena
provare attaccamento o aggrapparsi. Se riuscite a mantenere questa
capacità di calmare e indagare, sorgerà naturalmente la saggezza.
Qualunque cosa venga sperimentata, percepita, allora andrà a finire
sotto queste tre categorie: impermanenza, insoddisfazione, non-sé.
Questa è la meditazione \emph{vipassana}. La mente è ormai tranquilla e quando
affiorano stati mentali impuri, cacciateli in uno di questi tre bidoni
dell'immondizia. Questa è l'essenza della \emph{vipassana}: ridurre tutto a
impermanenza, insoddisfazione, non-sé. Buono, cattivo, orribile o
comunque sia, buttatelo via. In breve, dal bel mezzo delle tre
caratteristiche universali fiorirà la comprensione e l'intuizione
profonda, anche se questa sarà ancora debole. A questo stadio iniziale
la saggezza è ancora fluttuante e debole, ma cercate di mantenere la
pratica in modo continuativo. E' difficile da rendere a parole, ma è un
po' come se qualcuno volesse conoscermi: dovrebbe venire a vivere qui.
Gradualmente con il contatto quotidiano arriveremmo a
conoscerci.

\Section{Rispetto per la tradizione}

E' tempo ormai che cominciamo a meditare. Meditare per capire, per
abbandonare, per lasciare andare e per trovare la pace.

Un tempo ero un monaco errante. Viaggiavo per incontrare i maestri e per
cercare la solitudine. Non andavo in giro a offrire discorsi di Dhamma.
Andavo ad ascoltare i discorsi di Dhamma dei più grandi maestri del
tempo. Non andavo da loro a insegnare. Ascoltavo tutti i consigli che
essi mi offrivano. Perfino quando monaci più giovani e inesperti
cercavano di dirmi cosa era il Dhamma, ascoltavo pazientemente.
Raramente discutevo il Dhamma. Non vedevo l'utilità di lasciarmi
coinvolgere in lunghe discussioni. Quando accettavo un insegnamento lo
interiorizzavo subito, direttamente, proprio dove sottolineava la
rinuncia e il lasciar andare. Quello che facevo lo facevo seguendo
rinuncia e lasciar andare. Non abbiamo bisogno di diventare esperti
delle scritture. Ogni giorno che passa diventiamo più vecchi e ogni
giorno siamo accecati da un miraggio, perdendo di vista la realtà.
Praticare il Dhamma è una cosa ben diversa che studiarlo.

Non critico nessuna delle molte tecniche e stili di meditazione. Nessuna
è sbagliata fintanto che ne comprendiamo lo scopo e il significato.
Però, secondo me, chiamarci meditanti buddhisti e non seguire
strettamente il codice monastico di disciplina (\emph{vinaya}) non
funziona. Perché? Perché cerchiamo di evitare una fase vitale del
Sentiero. Tralasciare la virtù, il \emph{samādhi} o la saggezza non va bene.
Alcuni potrebbero dirvi di non attaccarvi alla serenità della
meditazione \emph{samatha}: ``Non preoccuparti di \emph{samatha}; vai direttamente
alla pratica \emph{vipassana} di saggezza e intuizione profonda''. Secondo come
la vedo io, se cerchiamo di volgerci direttamente verso la \emph{vipassana},
troveremo che sarà impossibile arrivare alla fine del viaggio.

Non abbandonate la pratica e le tecniche di meditazione di eminenti
Maestri della Foresta, quali i venerabili Ajahn Sao, Mun, Taungrut e
Upali. La via che hanno insegnato, se la pratichiamo come essi hanno
fatto, è totalmente affidabile e vera. Se seguiamo le loro orme avremo
una chiara comprensione diretta in noi stessi. Ajahn Sao mantenne una
virtù impeccabile. Mai disse che avremmo potuto lasciarla da parte. Se
questi grandi Maestri della Foresta hanno raccomandato di praticare la
meditazione e la disciplina monastica in un certo modo, dobbiamo cercare
di seguirli, se non altro per rispetto verso di loro. Se hanno detto di
farlo, facciamolo. Se hanno detto di smetterla perché è sbagliato,
allora smettiamola. E lo facciamo perché abbiamo fede. Lo facciamo
mettendoci un'incrollabile sincerità e determinazione. Lo facciamo
finché vedremo il Dhamma nel nostro stesso cuore, fino a
che \emph{saremo} il Dhamma. Così hanno insegnato i Maestri della
Foresta. Di conseguenza i loro discepoli hanno sviluppato un gran
rispetto, ammirazione e affetto per loro, perché, seguendo le loro orme,
hanno visto ciò che i loro maestri videro.

Provateci. Fatelo proprio come ve l'ho indicato. Se lo fate veramente,
vedrete il Dhamma, sarete il Dhamma. Se cominciate veramente la ricerca
cosa mai potrà fermarvi? Le contaminazioni mentali potrete superarle se
le avvicinate con la giusta strategia: siate capaci di rinunciare, siate
parchi con le parole, accontentatevi di poco, e abbandonate tutte le
idee e le opinioni che provengono dall'arroganza e dall'egocentrismo.
Allora sarete in grado di ascoltare chiunque, anche se quello che dicono
è sbagliato. A maggior ragione sarete anche in grado di ascoltare
pazientemente coloro che dicono il vero. Esaminatevi in questo modo. Vi
assicuro che è possibile, se ci provate. Ma gli studiosi difficilmente
vengono a praticare il Dhamma. Ce ne sono alcuni, ma pochi. E' un
peccato. Il fatto che siate arrivati fino a qui, è già di per sé degno
di lode. Significa che avete forza interiore. Alcuni monasteri
promuovono solo lo studio. I monaci non fanno altro che studiare, in
continuazione, senza fine e non recidono ciò che va reciso. Studiano
soltanto la parola ``pace''. Ma se riuscite a stare immobili allora
scoprirete qualcosa di grande valore. E' così che dovete portare avanti
la vostra ricerca. E' una ricerca molto importante e completamente
immobile. Va diritto al cuore di ciò che avete letto. Ma se gli studiosi
non praticano la meditazione, la loro conoscenza mancherà di
comprensione. Solo quando mettono in pratica gli insegnamenti, quelle
cose che hanno studiato diventeranno allora chiare e vivide.

Perciò cominciate a praticare! Sviluppate questo tipo di comprensione.
Provate a stare nella foresta e a vivere a stare in una di queste
piccole capanne. Provare per un po' questo tipo di vita e verificarla da
voi stessi sarà di maggior beneficio che solo leggere libri. Poi potrete
discutere con voi stessi. Mentre osservate la mente è come se essa
lasciasse andare tutto e riposasse nel suo stato naturale. Quando da
questo stato naturale di immobilità sorgono increspature e ondeggiamenti
sotto forma di pensieri e concetti, vuol dire che si è avviato il
processo condizionante dei \emph{saṅkhāra}. Siate cauti e guardinghi nei
riguardi di questo processo di condizionamento. Quando viene smossa,
scacciata dal suo stato naturale, la pratica del Dhamma non va più nella
direzione giusta. Diventa o auto-indulgenza o auto-punizione. Proprio
così. E' questo che fa sorgere la rete dei condizionamenti mentali. Se
lo stato mentale è buono, il condizionamento sarà positivo. Se è
cattivo, il condizionamento sarà negativo. Tutto ciò ha origine nella
vostra stessa mente.

Vi posso proprio dire che osservare da vicino come lavora la mente è un
gran divertimento. Potrei parlare su questo argomento per tutta la
giornata. Quando riuscite a vedere il comportamento della mente, vedrete
anche come funzionano questi processi e come la mente subisce un
continuo lavaggio del cervello da parte delle impurità mentali. Io vedo
la mente solo come un unico punto. Gli stati psicologici sono ospiti che
vengono a visitare questo punto. Alcuni vengono per una qualche
richiesta, altri a intrattenersi in visita. Arrivano nella sala
d'aspetto. Esercitate la mente in modo da osservarli e conoscerli con
gli occhi di una vigile consapevolezza. E' così che vi prendete cura del
cuore e della mente. Quando un visitatore si presenta fategli cenno di
allontanarsi. Se li fate entrare, dove li accomodate? C'è un solo posto
e ci siete seduti voi. Passate tutta la giornata su quel punto.

Questa è l'incrollabile e ferma consapevolezza del Buddha che vigila e
protegge la mente. Siete seduti proprio lì. Fin da quando siete emersi
dal ventre di vostra madre, ogni visitatore che si è presentato è
arrivato proprio lì. Non importa quante volte vengano, vengono comunque
sempre allo stesso punto. Siccome li conosce tutti, la consapevolezza
del Buddha è seduta lì sola, ferma e incrollabile. I viaggiatori
arrivano cercando di influenzare in qualche modo la mente, di
condizionarla o agitarla. Quando riescono a coinvolgere la mente nei
loro problemi, sorgono gli stati psicologici. Qualunque sia il problema,
ovunque porti, lasciatelo andare, non ha alcuna importanza per voi.
Semplicemente riconoscete gli ospiti man mano che arrivano. Una volta
entrati si accorgeranno che c'è solo una sedia, e fino a che la occupate
voi non avranno un posto dove sedersi. Arrivano pensando di riempirvi le
orecchie di pettegolezzi, ma questa volta non c'è posto per loro. E la
prossima volta che ritornano troveranno che di nuovo non c'è una sedia
libera. Non importa quante volte questi visitatori importuni si faranno
vedere, essi incontreranno sempre la stessa persona seduta allo stesso
posto. Non vi siete mai mossi da quella sedia. Per quanto tempo
continueranno ad andare avanti? Semplicemente parlando con loro riuscite
a conoscerli benissimo. Tutte le cose e tutte le persone che avete
conosciuto da quando sperimentate il mondo, verranno a farvi visita. Per
vedere il Dhamma in modo completo basta osservarli ed esserne
consapevoli proprio lì. Discutete, osservate e contemplate per conto
vostro.

E' così che si discute il Dhamma. Io non so parlare di nient'altro.
Posso andare avanti a parlare in questo modo, ma alla fine non è altro
che parlare e ascoltare. Vi consiglierei perciò di andare a praticare
realmente.

\Section{Conoscere a fondo la meditazione}

Se guardate con i vostri occhi, avrete certe esperienze. C'è il Sentiero
che vi guida e vi dirige. Man mano che andate avanti, la situazione
cambia e dovete adattare il vostro approccio per rimediare ai problemi
che sorgono. Può passare un certo tempo prima che vediate un'indicazione
stradale chiara. Se volete intraprendere lo stesso Sentiero che io ho
percorso, il viaggio deve senz'altro aver luogo all'interno di voi, nel
vostro cuore. Altrimenti, incontrerete numerosi ostacoli.

E' lo stesso che sentire un suono. Il sentire è una cosa, il suono
un'altra, e noi siamo consci di questo senza mischiare le due cose.
Contiamo sulla natura affinché ci fornisca il materiale grezzo su cui
indagare per la ricerca della Verità. Poi la mente seziona e separa da
sé i fenomeni. Cioè la mente semplicemente non viene coinvolta. Quando
l'orecchio sente un suono osservate ciò che avviene nella mente e nel
cuore. Vi si impigliano, vengono intrappolati e trascinati via dal
suono? Si irritano? Perlomeno cercate di conoscere questo. Quando un
suono poi viene registrato, non disturberà più la mente. Stando qui,
prendiamo le cose più a portata di mano invece di quelle lontane. Anche
se volessimo sfuggire al suono, non potremmo farlo. L'unico modo per
sfuggirlo è esercitare la mente a rimanere ferma di fronte ad esso.
Mettetelo giù il suono. Anche se abbandoniamo il suono, possiamo udire
lo stesso. Sentiamo, ma lasciamo andare il suono, perché lo abbiamo già
messo giù. Non è che dobbiamo separare forzatamente il suono dalla
mente. Se ne separa automaticamente lei stessa, quando abbandoniamo e
lasciamo andare. Poi, anche se vogliamo attaccarci al suono, la mente
non lo può più fare. Perché, una volta compresa la vera natura delle
cose visibili, dei suoni, degli odori, dei sapori e del resto, e quando
il cuore vede chiaramente, allora le cose che riguardano i sensi, tutte
senza eccezione, ricadono sotto il dominio delle caratteristiche
universali di impermanenza, insoddisfazione e mancanza di un sé.

Ogni volta che si sente un suono va compreso nei termini di queste tre
caratteristiche universali. Ogni volta che c'è contatto sensoriale con
l'orecchio, noi sentiamo, ma è come se non sentissimo. Ciò non significa
che la mente non funziona più. La consapevolezza e la mente si
intersecano e si fondono per controllarsi a vicenda, sempre, senza
sosta. Quando la mente raggiunge questo livello di pratica, non ha
importanza che via sceglieremo per svolgere la nostra ricerca.
Coltiveremo l'indagine dei fenomeni - uno dei fattori essenziali di
illuminazione - e questa analisi proseguirà per conto suo seguendo il
proprio impulso.

Discutete il Dhamma con voi stessi. Fate in modo da districare e
liberare i sentimenti, i ricordi, le percezioni, i pensieri, le
intenzioni, la coscienza. Niente riuscirà a toccarli se lasciate che
continuino a svolgere le loro funzioni indisturbati. Per coloro che
dominano la propria mente, questo processo di riflessione e indagine
scorre automaticamente; non c'è bisogno di dirigerlo intenzionalmente.
Verso qualunque direzione la mente si volga, immediatamente è presente
la contemplazione.

Se la pratica del Dhamma tocca questi livelli, ci saranno anche altri
benefici collaterali. Durante la notte non russeremo, non parleremo nel
sonno, non digrigneremo i denti, e non ci gireremo continuamente nel
letto. Anche svegliandoci da una profonda dormita, non ci sentiremo
sonnolenti. Saremo pieni di energia e vigili come se fossimo sempre
rimasti svegli. Un tempo io russavo, ma da quando la mente sta sempre
sveglia e vigile, non russo più. Come si può russare da svegli? E' solo
il corpo che si ferma e dorme. La mente è completamente sveglia giorno e
notte, sempre. Questa è la pura e sublime consapevolezza del Buddha: di
Colui che Conosce, del Risvegliato, del Gaudioso, del perfettamente
Radiante. Questa chiara consapevolezza non dorme mai. L'energia si
auto-rigenera e mai si intorpidisce o impigrisce. A questo stadio
possiamo andare avanti senza dormire per due o tre giorni. Quando il
corpo dà segni di esaurimento, ci sediamo in meditazione e
immediatamente entriamo in un profondo \emph{samādhi} per cinque o dieci
minuti. Quando usciamo da questo stato ci sentiamo freschi e rinvigoriti
come se avessimo dormito tutta la notte. Se non abbiamo eccessive
preoccupazioni per il nostro corpo, il sonno ha un'importanza minima.
Prendiamo tutte le misure necessarie per curare il corpo, ma non
mettiamoci in ansia per le sue condizioni fisiche. Che segua le leggi di
natura. Non dobbiamo essere noi a dire al corpo cosa deve fare. Se lo
dice da solo. E' come se qualcuno ci stimolasse, ci spronasse a
sforzarci sempre di più. Anche se ci sentiamo pigri, è come se ci fosse
una voce che ci sprona continuamente a essere diligenti. A questo punto
è impossibile ristagnare, perché lo sforzo e il progresso hanno
acquisito un inarrestabile impulso. Controllate voi stessi. E' da
parecchio che studiate e imparate; ora è tempo di studiare e imparare su
di voi.

All'inizio della pratica del Dhamma è di vitale importanza ritirarsi in
isolamento. Quando viviamo da soli in isolamento ricordiamoci le parole
del Ven. Sariputta: ``L'isolamento fisico è causa e condizione per il
sorgere dell'isolamento mentale, di stati di profondo \emph{samādhi} liberi da
ogni contatto sensoriale esterno. Questo isolamento della mente è, a sua
volta, causa e condizione per l'isolamento dalle contaminazioni mentali,
e per l'illuminazione''. Eppure c'è ancora gente che dice che
l'isolamento non è importante: ``se il cuore è tranquillo non ha
importanza dove si sta''. E' vero, ma dovremmo considerare che
all'inizio è importante l'isolamento fisico in un ambiente adatto. Oggi
stesso o al più presto, cercate un cimitero solitario in una foresta
remota, lontana da ogni abitazione. Provate a vivere completamente da
soli. Oppure cercate una vetta maestosa che incuta timore. Andateci da
soli. D'accordo? Vi divertirete un sacco per tutta la notte. Solo allora
capirete da voi stessi. Ci fu un tempo che anch'io pensavo che
l'isolamento fisico non fosse poi così importante. Era quello che
pensavo, ma una volta che lo sperimentai veramente, ebbi modo di
riflettere su ciò che aveva detto il Buddha. Il Beato aveva raccomandato
ai suoi discepoli di praticare in luoghi remoti lontani dalla società
umana. Ciò costituisce la base per un isolamento interno della mente che
a sua volta porta al totale isolamento dalle contaminazioni.

Supponiamo che siate una persona con casa e famiglia. Che isolamento
potete avere? Quando tornate a casa, appena mettete piede sulla soglia,
venite bersagliati dalla confusione e dai problemi. Questo non è
isolamento fisico. Allora ve ne andate a fare un ritiro in un luogo
remoto e l'atmosfera qui sarà completamente diversa. E' necessario
comprendere l'importanza dell'isolamento fisico e della solitudine negli
stadi iniziali della pratica del Dhamma. Poi cercate un maestro di
meditazione che vi istruisca, che vi guidi, vi consigli e corregga quei
punti in cui la vostra comprensione è errata. Perché è proprio dove non
capite bene e sbagliate che credete di essere nel giusto. Una volta che
il maestro ve lo abbia spiegato, capite ciò che è sbagliato, e proprio
dove il maestro dice che vi eravate sbagliati, proprio lì voi pensavate
di essere nel giusto.

Per quanto ne so, c'è un certo numero di monaci buddhisti studiosi che
cercano e ricercano basandosi sulle scritture. Non c'è nessuna ragione
che ci impedisca di sperimentare. Quando è il momento di aprire i libri
e studiare, impariamo in quel modo. Ma quando è il momento di armarsi e
di buttarsi nella battaglia potremmo trovarci a combattere in un modo
che non corrisponde alla teoria. Se un guerriero va in battaglia e
combatte secondo quanto ha appreso dai libri, non potrà tener testa al
nemico. Quando il guerriero è sincero e la lotta è reale, deve lottare
in un modo che va oltre la teoria. E' proprio così. Le parole del Buddha
nelle scritture sono solo linee guida ed esempi da seguire; limitarsi a
studiarle potrebbe portarci a non dare la giusta importanza al lato
pratico.

La via dei Maestri della Foresta è la via della rinuncia. Su questo
Sentiero vi è solo rinuncia. Sradichiamo le opinioni che sorgono
dall'egocentrismo. Sradichiamo la stessa essenza del senso del sé. Vi
assicuro che questa pratica sarà una sfida radicale per voi: andrà
dritta all'essenza, ma per quanto difficile sia, non rinunciate ai
Maestri della Foresta e ai loro insegnamenti. Senza una guida adatta, la
mente e il \emph{samādhi} possono essere molto ingannevoli. Possono accadere
cose che ci sembravano impossibili. Mi sono sempre avvicinato a questi
fenomeni con cautela e attenzione. Quando ero un giovane monaco,
all'inizio della mia pratica nei primi anni, non potevo ancora aver
fiducia nella mia mente. Però man mano che acquisivo una considerevole
esperienza e potevo fidarmi del lavoro della mente, niente costituiva
più un problema. Anche se si presentavano strani fenomeni, li lasciavo
fare. Se sappiamo come funzionano queste cose, esse cessano da sole. E'
tutto cibo per la saggezza. Col passare del tempo ci troveremo
perfettamente a nostro agio.

In meditazione, cose che di solito non sono sbagliate possono invece
divenire sbagliate. Per esempio ci sediamo a gambe incrociate con la
determinazione e la risoluzione: ``Va bene, questa volta niente
compromessi. Concentrerò la mente. State a vedere''. Non c'è alcuna
possibilità che questo sistema funzioni. Ogni volta che ci provavo era
un fallimento. Ma ci piace fare gli spacconi. Per quanto mi risulta, la
meditazione va avanti con un suo ritmo. Molte sere, sedendomi per la
meditazione, pensavo ``Va bene, questa sera non mi muoverò di qui,
perlomeno fino all'una di mattina''. Già questo pensiero predisponeva un
\emph{kamma} negativo; infatti non passava tanto tempo che il corpo veniva
assalito da un'infinità di dolori, che mi opprimevano fino al punto da
pensare che stavo per morire. Però nei periodi in cui la meditazione
andava bene non ponevo limiti alla durata della seduta. Non mi dicevo
``alle 8 o 9 o 10'' o a un'ora qualsiasi, ma semplicemente stavo seduto,
continuando con fermezza, lasciando andare con equanimità. Non forzate
la meditazione. Non cercate di interpretare ciò che sta accadendo. Non
costringete il cuore a rispondere a impossibili richieste di entrare in
stato di \emph{samādhi}; altrimenti diventerà più agitato e imprevedibile del
solito. Lasciate che il cuore e la mente si rilassino, comodi e a loro
agio.

Lasciate che il respiro fluisca facilmente con un suo proprio ritmo, né
troppo corto né troppo lungo. Non cercate di trasformarlo in qualcosa di
speciale. Lasciate che il corpo si riposi, comodo e a suo agio. Poi
continuate. La mente vi chiederà: ``Fino a che ora mediteremo stasera? A
che ora smetteremo?'' Brontola senza sosta, per cui dovete rimproverarla
aspramente, ``Ehi ragazza mia, lasciami in pace, smettila''. Questa
intrigante che non fa che porre domande va regolarmente messa a tacere,
perché non sono altro che le contaminazioni che vengono a disturbarvi.
Non prestate loro attenzione. Dovete essere duri: ``Che io smetta subito
o vada avanti tutta la notte, non sono affari tuoi! Se voglio rimanere
seduto tutta la notte non deve importare a nessuno, perciò perché vieni
qui a mettere il naso nei miei affari di meditazione?'' Dovete cacciare
via quell'impicciona. Poi potete continuare a meditare quanto volete,
secondo quello che ritenete giusto.

Quando permettete alla mente di rilassarsi ed essere a suo agio,
diventerà calma. Facendo questo tipo di esperienza, sapete allora
riconoscere e valutare il potere dell'attaccamento. Quando riuscirete a
stare seduti a lungo, molto a lungo, oltre la mezzanotte, sempre comodi
e rilassati, allora vuol dire che state diventando padroni della vostra
meditazione. Capirete che veramente l'attaccamento contamina la mente.
Alcuni, quando si siedono a meditare, accendono un bastoncino d'incenso
e giurano a se stessi ``Non mi alzerò fino a quando questo bastoncino
d'incenso non sarà finito''. Poi si siedono. Dopo un tempo che a loro
pare un'ora aprono gli occhi e realizzano che sono passati solo cinque
minuti. Guardano l'incenso, delusi da quanto sia ancora lungo il
bastoncino. Chiudono gli occhi e continuano. Ma presto gli occhi si
aprono di nuovo a controllare il bastoncino d'incenso. Gente che medita
così non arriva da nessuna parte. Non fatelo. Se vi sedete e cominciate
a pensare a quel pezzetto d'incenso - ``Mi chiedo se sarà finalmente
finito'' - la meditazione non va avanti. Non date importanza a cose del
genere. La mente non deve fare niente di speciale.

Se ci poniamo il compito di sviluppare la mente con la meditazione, non
permettete all'avidità inquinante di conoscere le regole del gioco o il
vostro scopo. ``Come mediterai ora, Venerabile?'' domanda. ``Per quanto
ne avrai? Fino a che ora pensi di andare avanti?'' L'avidità continua a
imperversare fino a che ci arrendiamo e arriviamo a un accordo. Una
volta che diciamo che staremo seduti fino a mezzanotte, quella comincerà
a tormentarci. Prima ancora che sia passata un'ora ci sentiremo così
agitati e impazienti da non poter continuare. Altri impedimenti ci
assaliranno, proprio mentre ci rimproveriamo. ``Ma dai! Pensi che
questa seduta ti ucciderà? Hai detto che volevi passare la mente nel
\emph{samādhi}, e invece è ancora instabile e gira a vuoto. Hai fatto una
promessa ma non l'hai mantenuta''. Sono pensieri di sconforto e di
frustrazione che assalgono la mente e ci sprofondano in un mare di
auto-accuse. Non c'è nessun altro da rimproverare o con cui arrabbiarsi
e questo rende tutto più difficile. Una volta fatto un giuramento
dobbiamo mantenerlo. O lo manteniamo o moriamo. Se facciamo il
giuramento di sedere per un certo tempo, non dovremmo poi infrangere la
promessa e smettere. Nel frattempo però praticate e maturate in modo
graduale. Non c'è nessun bisogno di fare voti sensazionali. Cercate
invece di allenare la mente con fermezza e costanza. Di tanto in tanto
avrete una meditazione tranquilla, e spariranno tutti i dolori e i
disagi del corpo. Il dolore alle caviglie e alle ginocchia smetterà da
solo.

Se, mentre proviamo a coltivare la meditazione, cominciano a sorgere
strane immagini, visioni o percezioni sensoriali, la prima cosa da fare
è controllare lo stato della nostra mente. Non omettete questo passo
essenziale. La mente deve essere relativamente tranquilla perché possano
sorgere queste immagini. Non desiderate che appaiano e non desiderate
che non appaiano. Se sorgono, esaminatele, ma non permettete loro di
ingannarvi. Ricordatevi che non sono vostre. Sono impermanenti,
insoddisfacenti e prive di un sé, proprio come qualsiasi altra cosa.
Anche se fossero reali non fermatevi su di esse, prestandovi troppa
attenzione. Se si rifiutano ostinatamente di sparire, allora riportate
l'attenzione sul respiro con maggior vigore. Prendete tre lunghi respiri
e ogni volta esalate liberando completamente i polmoni. Questo può
risolvere la cosa. Cercate di rifocalizzare l'attenzione.

Non impossessatevi di questi fenomeni. Non sono altro che quello che
sono, e ciò che sono provoca potenzialmente un'illusione. O ci piacciono
e ce ne innamoriamo, oppure la mente si intossica di paura. Sono
inaffidabili: possono non essere veri o possono non essere affatto
quello che sembrano. Se li sperimentate non cercate di interpretarne il
significato o proiettare un significato in essi. Ricordatevi che non
sono voi, perciò non correte dietro a queste visioni o sensazioni.
Invece, andate subito a controllare lo stato presente della mente.
Questa è la nostra regola pratica. Se non teniamo conto di questo
principio basilare e ci lasciamo trascinare in ciò che crediamo di
vedere, va a finire che ci dimentichiamo di noi stessi, cominciamo a
parlare a vanvera o anche a dare i numeri. Possiamo perdere la bussola
fino al punto da non poterci più relazionare con gli altri a un livello
normale. Confidate nel cuore. Qualsiasi cosa accada continuate ad
osservare il cuore e la mente. Esperienze meditative strane possono
essere benefiche per coloro che hanno saggezza, ma pericolose per quelli
che non ce l'hanno. Qualsiasi cosa avvenga non esaltatevi né
allarmatevi. Se ci sono esperienze particolari, ci sono e basta.

Un altro modo di praticare il Dhamma è quello di contemplare ed
esaminare tutto ciò che vediamo, facciamo e sperimentiamo. La
meditazione non ha mai fine. Alcuni credono che quando hanno finito le
sessioni di meditazione seduta o camminata, bisogna smettere e
riposarsi. Smettono di concentrare la mente sull'oggetto di meditazione
o sul tema di contemplazione. Li lasciano perdere completamente. Non
praticate così. Indagate su ogni cosa che vedete per capire come è
realmente. Contemplate la buona gente del mondo. Contemplate anche
quella cattiva. Osservate profondamente il ricco e il potente; il povero
e il reietto. Quando vedete un bambino, una persona anziana, un giovane
o una giovane, indagate sul significato dell'età. Tutto è materiale di
indagine. E' così che coltivate la mente. La contemplazione che porta al
Dhamma è la contemplazione della condizionalità, del processo di causa
ed effetto, in tutte le sue manifestazioni: maggiore o minore, bianco o
nero, buono o cattivo. In breve, tutto. Quando avete un pensiero,
riconoscetelo come un pensiero e contemplate che è solo quello, niente
di più. Tutte queste cose vanno a finire nel cimitero dell'impermanenza,
dell'insoddisfazione e del non-sé, per cui non attaccatevi morbosamente
a nessuna di esse. E' il cimitero di tutti i fenomeni. Seppelliteli o
cremateli per poter sperimentare la Verità.

Avere un'intuizione profonda nell'impermanenza vuol dire non lasciarsi
andare alla sofferenza. Bisogna indagare con saggezza. Per esempio,
otteniamo qualcosa che riteniamo buono o piacevole e perciò ne siamo
felici. Osservate da vicino e a lungo questa cosiddetta bontà e
piacevolezza. Certe volte, dopo un po' che l'abbiamo, ce ne stufiamo.
Vogliamo dar via l'oggetto che l'ha procurata oppure venderlo. Se non
c'è nessuno disposto a comprarlo, siamo pronti a buttarlo via. Perché?
Cosa c'è dietro a questo modo di fare? Tutto è impermanente, incostante,
mutevole, ecco il perché. Se non possiamo venderlo o addirittura nemmeno
gettarlo, cominciamo a soffrire. Tutto ruota intorno a questo. Ma una
volta che abbiamo compreso perfettamente un tale evento, quando
sorgeranno altre situazioni simili, comprenderemo che sono la stessa
cosa. Questo è semplicemente il modo in cui sono le cose. Come si suol
dire ``Quando ne avete visto uno, li avete visti tutti''.

Certe volte vediamo cose che non ci piacciono. Altre volte sentiamo
rumori spiacevoli che ci disturbano e perciò ci irritiamo. Esaminate
tutto ciò e ricordatevelo. Perché forse in un prossimo futuro potrà
capitare che ci piaceranno quegli stessi rumori che oggi ci disturbano.
Potremmo addirittura deliziarci di quello che un tempo detestavamo. E'
possibile! Allora, in un lampo di chiarezza e intuizione profonda,
capiremo ``Aha! Tutto è impermanente, incapace di soddisfare
completamente, e senza un sé.''. Buttateli nella tomba comune delle tre
caratteristiche universali. E allora cesserà l'attaccamento a ciò che è
piacevole, a ciò che possediamo, a ciò che siamo. Giungeremo a vedere
che tutto è fondamentalmente la stessa cosa. Allora ogni cosa che
vediamo genererà una visione profonda del Dhamma.

Ho detto tutto ciò perché voi possiate ascoltare e pensarci sopra. E'
una chiacchierata, e basta. Quando la gente viene a vedermi, io parlo.
Questo argomento non è tale da doversi sedere in circolo e parlarne per
ore. \emph{Fatelo e basta}. Comprendetelo e fatelo. E' come quando
chiamiamo un amico per andare insieme in qualche posto. Lo invitiamo. Ne
riceviamo una risposta. Poi usciamo senza farne un problema. Diciamo
solo quello che va detto e basta. Sulla meditazione io vi posso dire due
o tre cose perché l'ho praticata. Ma può darsi che sbagli, sapete. Il
vostro compito è quello di indagare e scoprire voi stessi se ciò che ho
detto è vero.

